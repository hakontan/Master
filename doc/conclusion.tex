\chapter{Concluding remarks}\label{sec:conclusion}
In this thesis, I have derived and tested a simple model for modelling the velocity of matter around cosmic filaments using linear theory and compared it to data from numerical simulations. This experiment is interesting as it can potentially allow for an additional set of experiments for extracting valuable cosmological information from the density field. I have also tested a proposed improvement to the radial velocity of matter around voids and applied it to a model of the void-galaxy cross-correlation function using a linear bias approximation. This was compared to the regular model for the radial velocity derived from linear theory. The analysis conducted on voids is useful as future observational datasets will grow in size. This will in turn require a better understanding of potential errors and improvements for current methodology. For these velocity models, the quadrupole of the void-galaxy cross-correlation was compared by applying cuts to the dataset including voids of different sizes in order to see if this affected the validity of the velocity approximations. Lastly, using a dark matter overdensity profile from another dataset, the regular velocity model was tested to see if potential corrections could be applied to an arbitrary density profile while still providing accurate parameter fits.
\section{Summary of main results}\label{sec:mainresults}
Using linear theory, a simple model for the radial velocity perpendicular to the spine of cosmological filaments, similar to the model already tested on voids, was derived and tested. Overall this model looked promising, however, the comparison with numerical data illustrated the presence of irregularities regarding the amplitude of the velocity profile close to the filament spine. The amplitude of the stacked radial velocity profile for filaments showed large deviations depending on the dataset and $\sigma$-threshold chosen as input to the DisPerSE code. The promising aspect of the model is that it accurately replicated the shape of the numerically calculated velocity profile. The modelling of the radial velocity profile for filaments is reliant on accurate modelling of the density profile. The density profile for the filaments showed varying densities close to the filament spine depending on the dataset and $\sigma$-threshold chosen. This may be an effect of how DisPerSE applies cuts to the filament catalogue removing filaments of different lengths, which in turn may affect the number density in the stacked density profile. It may also be a reasonable result, and in turn, will suggest that linear theory does not provide a reasonable approximation to the radial velocity of matter around filaments. However, on the basis of the results presented in this thesis, although the results looked promising, one is not able to draw a conclusion of whether or not linear theory is a sufficient approximation for the behavior of matter around filaments.\\\indent

In this thesis, the regular linear model for the radial velocity of matter around filaments was also tested. In addition, a proposed improvement to the regular velocity model was tested and both models were compared with numerical data. Both of these velocity models were implemented into the anisotropic void-galaxy cross-correlation function and their performance was tested in terms of fitting cosmological parameters. When conducting the comparison between the two parameters, different cuts to the dataset was applied. The $\epsilon$ parameter, in which directly linked to the cosmological parameters, received consistently accurate fits using the regular velocity model. The model with the velocity correction term was biased in the sense that it consistently underestimated fits to $\epsilon$. On the other hand, the model with the correction term made better fits to the $\beta$ parameter when small voids were considered. However, when only large voids were considered, the regular model consistently provided the best fits for both of these parameters for the two largest datasets. The modelling of the velocity itself showed that when applying cuts only keeping the larger voids, the regular model qualitatively resembled the numerically calculated radial velocity in an accurate fashion. When applying cuts including only smaller voids there was not as prominent of an all round difference, but both models seemed to capture certain features better. The regular model performed better at capturing the velocity on large scales, while the amplitude predicted by the model with correction was more accurate with these cuts. These results suggest that when applying cuts to the dataset, excluding small voids which may be subject to non-linear behavior, the regular velocity model performed better.\\\indent
Finally, the model for the anisotropic cross-correlation function between voids and galaxies was tested using a dark matter overdensity profile gathered from another dataset. Using this overdensity profile only the regular velocity model was implemented. For this test, the cuts only considering smaller voids was excluded. Across all datasets and cuts, this model managed to provide accurate samples for the $\epsilon$ parameter. Using no cuts, the model provided unsatisfying samples for $f\sigma_8$. This improved however when cuts only considering larger voids was considered, but the sampling was still not accurate enough for this implementation of the model to be deemed sufficient.  This suggests that the model was not able to sample the amplitude of dark matter density profile accurately. The scaling of the overdensity profile, parametrized by the $r_{scale}$ parameter, did however show that the model captured in which direction the density profile should be shifted. Overall, this analysis suggests that when sampling the $\epsilon$ parameter, using an arbitrary dark matter overdensity profile and correcting it may be sufficient. However, if one wants to sample $f\sigma_8$, the methodology applied in this analysis may not be sufficient.
\section{Suggestions for future work}\label{sec:futurework}
Although this thesis showed promising results when modelling the radial velocity of matter around filaments, further research is needed in order to conclude whether the approach proposed in this thesis may prove to be a full-fledged addition to LSS studies. The results showed that changing $\sigma$-threshold as input to the DisPerSE code effectively lowered the amount of shorter filaments. This also affected the amplitude of the radial velocity predicted by the model. In my analysis, I chose to apply cuts including filaments with length $20$Mpc/h$<l<100$Mpc/h. This is a rather generous cut and applying more cuts, similar to what was done when studying voids, could indicate whether modelling the radial velocity using linear theory is a better approximation for filaments of different length. The model was also only tested on the smaller MD1 and MD2 datasets. Applying the model to the larger MD3 or MD4 datasets could also prove valuable. However, the method itself for calculating the numerical stacked velocity is computationally heavy when the number of particles increases. This is partly due to an increasing amount of cross products needed to gain the radial component of the velocity vector. If one is to judge whether the model sufficiently captures the behavior of the stacked radial velocity then this also has to be calculated. It could therefore be useful to implement the model in a more efficient manner than a pure Python implementation. The velocity profiles also showed a feature resembling a hook. As discussed, this feature may be attributed to most of the matter around filaments being contained at intersections where filaments meet. In order to test whether this is the case, a 2D modelling of the velocity field around filaments would be an interesting addition to the 1D radial velocity. A natural step once the modelling of the radial velocity has been tested, and in case it is deemed to be sufficient approximation, is to derive an expression for the filament-galaxy cross-correlation function in redshift space, similar to what already has been done for the void-galaxy cross-correlation function. This can potentially open up for future analysis similar to the what is conducted on voids in this thesis. \\\indent

The results for the void galaxy cross-correlation function suggested that the preferable way to model the radial velocity of matter around voids was the regular model derived from linear theory. However, using a linear bias approximation, this model did not manage to accurately sample the $\beta$ parameter when small voids were included in the catalogue. The sampling did however improve when cuts were applied. The $r>40$Mpc/h cut however could be viewed as rather conservative. Further effort could also be invested into applying cuts including smaller voids and try to better understand for what scales linear-theory is still a good approximation. This will increase the sample size of void-galaxy pairs and will in turn reduce noise in the numerically calculated cross-correlation function. The model with the velocity correction did however fit the $\beta$ parameter better on smaller voids. Adding an extra dimensionless nuisance parameter as a scaling for the correction term could be a reasonable improvement to this particular model.\\\indent
The model with the dark matter overdensity profile managed to provide satisfying fits the $\epsilon$ parameter across all datasets and cuts. It did not manage to sample $f\sigma_8$. Applying cuts, however, did make the sampling of the parameter better. Applying the $r>40$Mpc/h cut did reduce the number of void-halo pairs and introduced noise to the cross-correlation function. A future suggestion could be to use even more conservative cuts. However, in order to do this, a larger dataset may be necessary in order to increase the samples of void-halo pairs. Performing the same analysis with the complete information about $\delta_{dm}$ corresponding to the given dataset could also prove valuable. A natural next step for all the methods tested, including the analysis of filaments, would be to use the redshifted data catalogue and apply the reconstruction algorithm and see how this affects the model results.