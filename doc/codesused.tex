\chapter{Codes utilized}
\section{DisPerSE-"Discrete Persistent Structures Extractor"}
In this work identification of the filamentary structure of the cosmic web in
data sets is crucial. To achieve this the code
DisPerSE\cite{2011MNRAS.414..350S}\cite{2011MNRAS.414..384S} is utilized.
DisPerSE is a code that was developed to identify topological features of the
cosmic web in cosmological particle distributions. DisPerSE utilized discrete
Morse Theory to identify the topological structures of the cosmic web in data
sets. In this section i will give a brief intuitive overview of how DisPerSE works to
identify the filamentary structure from a particle distribution. For a thorough
review i recommend the paper \cite{2011MNRAS.414..350S}, in which the following
chapter is based on, describing the theory behind DisPerSE and the
references therein.
\subsection{Morse-Theory}
To identify the features of the cosmic web from the density field of our
particle distribution, DisPerSE relies on Morse Theory \cite{Morse}. Morse
theory is a mathematical framework used to relate the geometrical and
topological properties of a function. In our case we utilize morse theory to
study the relation between the density field and topological features such as
filaments, voids and walls.
This is done by studying the gradient of a smooth function $f$. By studying the
gradient of the said function $f$, one can analyze the topological features of
the manifold. In general, to apply Morse theory one needs a smooth scalar function $f$
that is twice differentiable. In our case this scalar function is the density
field of our simulation volume. In Morse theory the gradient of the function $f$
is defined as $\nabla_xf(\vec{x})=df(\vec{x})/d\vec{x}$. This function
specifies the direction of steepest ascent for our scalar density function. The
points where $\nabla_xf=0$ are called critical points. These critical
points can be classified by studying the Hessian matrix of $f$ given as
\begin{equation}
    \mathcal{H}_f(\vec{x})=d^2f(\vec{x})/dx_idx_j.
\end{equation}
The critical points are classified by studying the eigenvalues of the Hessian
matrix. A critical point is of order $k$ when the Hessian matrix has exactly $k$
negative eigenvalues. This way one can classify the critical points as maxima
with $k=2$, saddle point with $k=1$ and a minima with $k=0$. This analysis also
implies that for morse theory to be applicable, the function $f$ has to satisfy
the condition $\mathcal{H}_f(\vec{x})\neq 0$ where $\nabla_xf=0$. From this it
follows that any function that satisfies this condition is called a Morse
function.\\
 
It is previously stated that the gradient points in a preferred direction for any non-critical
point. From the gradient one can therefore define integral lines between the critical
points. An integral lines is a parametrised curve $L(t)\in\mathbb{R}^n$ that satisfies
\begin{equation}
    \frac{dL(t)}{dt}=\nabla_xf(\vec{x}).
\end{equation}
These integral lines will always have critical points as their origin and
destination. An important property of these integral lines is that they cover
all of $\mathbb{R}^d$, but they never intersect eachother. They can however
share the same origin or destination. With these integral lines one can define
ascending and descending n-manifolds. If one consider a critical point $P$ of
order $k$ on the Morse function living in $\mathbb{R}^d$, we have an ascending manifold of
$d-k$ dimensions. This ascending manifold is constructed by the set of points reached by
all integral lines with origin at the critical point $P$. The descending
manifold is a $k-$dimensional region of space defined by all integral lines with
destination at point $P$. This leads to the definition of the Morse-complex.
The Morse complex is simply the set of all the ascending or descending
manifolds.\\
\subsection{Discrete Morse-theory}
The formality currently introduced deals with smooth continuous functions. This
however is rarely applicable to measured data or simulations. For an example the density field
of our simulation volume is derived from a discrete set of
point particles. Therefore discrete
Morse-theory\cite{FORMAN199890} is applied as to utilize Morse-theory on
discrete data. Instead of working on smooth functions discrete Morse-theory is
applied to what is called a simplicial complex. A simplicial complex is a space made up
of simplices. A simplex is a generalization of
a triangle to an arbitrary number of dimensions. For example a k-dimensional
simplex, reffered to as k-simplex, is represented as
\begin{itemize}
    \item a $0$-simplex is a point
    \item a $1$-simplex is a line segment
    \item a $2$-simplex is a triangle
    \item a $3$-simplex is a tetrahedron/pyramid etc.
\end{itemize}
A k-simplex will be denoted as $\sigma_k$ defined by the set of points
$\sigma_k=\{p_0, \dots,p_k\}$. Similarily one can define the face of a simplex
to be a subset of the original k-simplex. We can define the face as an
l-simplex $\gamma_l=\{p_0,\dots,p_l\}$ with $l\leq k$. If $\gamma_l$ is a face
of $\sigma_k$, then $\sigma_k$ is called a coface of $\gamma_l$. When $k$ and
$l$ only differs by one, a face is called a facet and a coface is called a cofacet. These simplices make up
what is called a simplicial complex in which a discrete Morse-function is
defined and discrete Morse-theory is applied. A simplicial complex $K$ are made
up of a finite set of simplices such that $\sigma_k\in K$. With this one can
define a discrete Morse-function $f$ as a function that maps a real value
$f(\sigma_k)$ for each simplex in the simplicial complex $K$. As for the
Morse-function defined in. regular Morse-theory, the discrete counterpart also has
to be differentiable and the gradient can only flow towards one preferred direction
locally. For this to occur and a Morse-function to be defined on the simplicial
complex it has to satisfy the following criteria
\begin{itemize}
    \item there exists at most one facet $\xi_{k-1}$ of $\sigma_{k}$ such
    that $f(\sigma_k)\leq f(\xi_{k-1})$, and
    \item there exists at most one cofacet $\chi_{k+1}$ of $\sigma_{k}$ such
    that $f(\sigma_k)\geq f(\chi_{k+1})$.
\end{itemize}
This criteria states that locally a simplex has a lower value than its facet and
a higher value than its cofacet.




\subsection{Topological Persistence}

\section{REVOLVER-REal-space VOid Locations from surVEy Reconstruction}