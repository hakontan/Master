\chapter{Results}
Kapittel med midlertidige resultater for å prøve å danne en rød tråd.
\section{Dataset}
\section{Void analysis}
\subsection{Density profiles}
\begin{figure}[htbp]\label{fig:histMD1}
    \subfigure[]{\includegraphics[width=0.5\textwidth]{figures/Density_profiles/Voids/density_profileMD1_all.pdf}}\hspace{1em}%
    \subfigure[]{\includegraphics[width=0.5\textwidth]{figures/Density_profiles/Voids/density_profileMD1_0_50.pdf}}
    \subfigure[]{\includegraphics[width=0.5\textwidth]{figures/Density_profiles/Voids/density_profileMD1_60_80.pdf}}\hspace{1em}%
    \subfigure[]{\includegraphics[width=0.5\textwidth]{figures/Density_profiles/Voids/density_profileMD1_80_infty.pdf}}
    \caption{Figure showing the density profile for voids found in the Multidark 1 dataset. Here the radius of voids considered are in the radius range of: all voids (a), $0$Mpc/h $\leq r\leq 50$Mpc/h (b), $60$Mpc/h $\leq r\leq 80$Mpc/h (c) and $r\geq 80$Mpc/h (d).}
\end{figure}

\begin{figure}[htbp]\label{fig:histMD1}
    \subfigure[]{\includegraphics[width=0.5\textwidth]{figures/Density_profiles/Voids/density_profileMD2_all.pdf}}\hspace{1em}%
    \subfigure[]{\includegraphics[width=0.5\textwidth]{figures/Density_profiles/Voids/density_profileMD2_0_50.pdf}}
    \subfigure[]{\includegraphics[width=0.5\textwidth]{figures/Density_profiles/Voids/density_profileMD2_60_80.pdf}}\hspace{1em}%
    \subfigure[]{\includegraphics[width=0.5\textwidth]{figures/Density_profiles/Voids/density_profileMD2_80_infty.pdf}}
    \caption{Figure showing the density profile for voids found in the Multidark 2 dataset. Here the radius of voids considered are in the radius range of: all voids (a), $0$Mpc/h $\leq r\leq 50$Mpc/h (b), $60$Mpc/h $\leq r\leq 80$Mpc/h (c) and $r\geq 80$Mpc/h (d).}
\end{figure}
\subsection{Radial velocity profiles}
\begin{figure}[htbp]\label{fig:histMD1}
    \subfigure[]{\includegraphics[width=0.5\textwidth]{figures/Theory_vs_v_r/Voids/velocity_profileMD1_all.pdf}}\hspace{1em}%
    \subfigure[]{\includegraphics[width=0.5\textwidth]{figures/Theory_vs_v_r/Voids/velocity_profileMD1_0_50.pdf}}
    \subfigure[]{\includegraphics[width=0.5\textwidth]{figures/Theory_vs_v_r/Voids/velocity_profileMD1_60_80.pdf}}\hspace{1em}%
    \subfigure[]{\includegraphics[width=0.5\textwidth]{figures/Theory_vs_v_r/Voids/velocity_profileMD1_80_infty.pdf}}
    \caption{Figure showing the velocity profile for voids found in the Multidark 1 dataset. Here the radius of voids considered are in the radius range of: all voids (a), $0$Mpc/h $\leq r\leq 50$Mpc/h (b), $60$Mpc/h $\leq r\leq 80$Mpc/h (c) and $r\geq 80$Mpc/h (d).}
\end{figure}

\begin{figure}[htbp]\label{fig:histMD1}
    \subfigure[]{\includegraphics[width=0.5\textwidth]{figures/Theory_vs_v_r/Voids/velocity_profileMD2_all.pdf}}\hspace{1em}%
    \subfigure[]{\includegraphics[width=0.5\textwidth]{figures/Theory_vs_v_r/Voids/velocity_profileMD2_0_50.pdf}}
    \subfigure[]{\includegraphics[width=0.5\textwidth]{figures/Theory_vs_v_r/Voids/velocity_profileMD2_60_80.pdf}}\hspace{1em}%
    \subfigure[]{\includegraphics[width=0.5\textwidth]{figures/Theory_vs_v_r/Voids/velocity_profileMD2_80_infty.pdf}}
    \caption{Figure showing the velocity profile for voids found in the Multidark 2 dataset. Here the radius of voids considered are in the radius range of: all voids (a), $0$Mpc/h $\leq r\leq 50$Mpc/h (b), $60$Mpc/h $\leq r\leq 80$Mpc/h (c) and $r\geq 80$Mpc/h (d).}
\end{figure}
\subsection{Correlation function void-galaxy $\xi_{vg}$}

\section{Filament analysis}
\subsection{Effect of choosing persistence ratio on dataset}
As discussed in section \ref{sec:persistence} DisPerSE has an option to filter
noise through setting a persistence threshold. This has a profound effect on the
results.
\begin{figure}[htbp]\label{fig:scatterMD1}
    \subfigure[]{\includegraphics[width=0.5\textwidth]{figures/scatterplots/scatter_MD1_all.pdf}}\hspace{1em}%
    \subfigure[]{\includegraphics[width=0.5\textwidth]{figures/scatterplots/scatter_MD1_s1.pdf}}
    \subfigure[]{\includegraphics[width=0.5\textwidth]{figures/scatterplots/scatter_MD1_s2.pdf}}\hspace{1em}%
    \subfigure[]{\includegraphics[width=0.5\textwidth]{figures/scatterplots/scatter_MD1_s3.pdf}}
    \caption{Figure showing a slice of the multidark 1 dataset for $950$Mpc/h$\leq z\leq1050$Mpc/h in the $x$, $y$ plane. Here the persistance threshold given to DisPerSE is varied between no cuts (a), $\sigma=1$ (b), $\sigma=2$ (c), $\sigma=3$ (d).}
\end{figure}
\begin{figure}[htbp]\label{fig:histMD1}
    \subfigure[]{\includegraphics[width=0.5\textwidth]{figures/histograms/filament_histMD1_all.pdf}}\hspace{1em}%
    \subfigure[]{\includegraphics[width=0.5\textwidth]{figures/histograms/filament_histMD1_s1.pdf}}
    \subfigure[]{\includegraphics[width=0.5\textwidth]{figures/histograms/filament_histMD1_s2.pdf}}\hspace{1em}%
    \subfigure[]{\includegraphics[width=0.5\textwidth]{figures/histograms/filament_histMD1_s3.pdf}}
    \caption{Figure showing histograms of filament lengths calculated by disperse the multidark 1 dataset for $950$Mpc/h$\leq z\leq1050$Mpc/h in the $x$, $y$ plane. Here the persistance threshold given to DisPerSE is varied between no cuts (a), $\sigma=1$ (b), $\sigma=2$ (c), $\sigma=3$ (d). Each blue dot represents a halo particle while all lines represents filaments assigned by DisPerSE.}
\end{figure}
Figures \ref{fig:scatterMD1} and \ref{fig:histMD1} shows scatter plots and
histograms of filament lengths for the Multidark 1 dataset containing $534559$
halo particles.
\begin{figure}[htbp]
    \subfigure[]{\includegraphics[width=0.5\textwidth]{figures/scatterplots/scatter_MD2_all.pdf}}\hspace{1em}%
    \subfigure[]{\includegraphics[width=0.5\textwidth]{figures/scatterplots/scatter_MD2_s1.pdf}}
    \subfigure[]{\includegraphics[width=0.5\textwidth]{figures/scatterplots/scatter_MD2_s2.pdf}}\hspace{1em}%
    \subfigure[]{\includegraphics[width=0.5\textwidth]{figures/scatterplots/scatter_MD2_s3.pdf}}
    \caption{Figure showing a slice of the multidark 2 dataset for $975$Mpc/h $\leq z\leq1025$Mpc/h in the $x$, $y$ plane. Here the persistance threshold given to DisPerSE is varied between no cuts (a), $\sigma=1$ (b), $\sigma=2$ (c), $\sigma=3$ (d). Each blue dot represents a halo particle while all lines represents filaments assigned by DisPerSE.}
\end{figure}
\begin{figure}[htbp]
    \subfigure[]{\includegraphics[width=0.5\textwidth]{figures/histograms/filament_histMD2_all.pdf}}\hspace{1em}%
    \subfigure[]{\includegraphics[width=0.5\textwidth]{figures/histograms/filament_histMD2_s1.pdf}}
    \subfigure[]{\includegraphics[width=0.5\textwidth]{figures/histograms/filament_histMD2_s2.pdf}}\hspace{1em}%
    \subfigure[]{\includegraphics[width=0.5\textwidth]{figures/histograms/filament_histMD2_s3.pdf}}
    \caption{Figure showing histograms of filament lengths calculated by disperse the multidark 2 dataset for $975$Mpc/h $\leq z\leq1025$Mpc/h in the $x$, $y$ plane. Here the persistance threshold given to DisPerSE is varied between no cuts (a), $\sigma=1$ (b), $\sigma=2$ (c), $\sigma=3$ (d).}
\end{figure}
\subsection{Density profiles}
\subsection{Radial velocity profiles}
\subsection{Correlation function filament-galaxy $\xi_{fg}$}
