\chapter{Results}

\section{Filament analysis}

\subsection{Effect of choosing persistence ratio on dataset}
As discussed in section \ref{sec:persistence} DisPerSE has an option to filter
noise through setting a persistence threshold. This has a profound effect on the
results.
\begin{figure}[htbp]\label{fig:scatterMD1}
    \subfigure[]{\includegraphics[width=0.5\textwidth]{figures/scatterplots/scatter_MD1_all.pdf}}\hspace{1em}%
    \subfigure[]{\includegraphics[width=0.5\textwidth]{figures/scatterplots/scatter_MD1_s1.pdf}}
    \subfigure[]{\includegraphics[width=0.5\textwidth]{figures/scatterplots/scatter_MD1_s2.pdf}}\hspace{1em}%
    \subfigure[]{\includegraphics[width=0.5\textwidth]{figures/scatterplots/scatter_MD1_s3.pdf}}
    \caption{Figure showing a slice of the multidark 1 dataset for $950$Mpc/h$\leq z\leq1050$Mpc/h in the $x$, $y$ plane. Here the persistance threshold given to DisPerSE is varied between no cuts (a), $\sigma=1$ (b), $\sigma=2$ (c), $\sigma=3$ (d).}
\end{figure}
\begin{figure}[htbp]\label{fig:histMD1}
    \subfigure[]{\includegraphics[width=0.5\textwidth]{figures/histograms/filament_histMD1_all.pdf}}\hspace{1em}%
    \subfigure[]{\includegraphics[width=0.5\textwidth]{figures/histograms/filament_histMD1_s1.pdf}}
    \subfigure[]{\includegraphics[width=0.5\textwidth]{figures/histograms/filament_histMD1_s2.pdf}}\hspace{1em}%
    \subfigure[]{\includegraphics[width=0.5\textwidth]{figures/histograms/filament_histMD1_s3.pdf}}
    \caption{Figure showing histograms of filament lengths calculated by disperse the multidark 1 dataset for $950$Mpc/h$\leq z\leq1050$Mpc/h in the $x$, $y$ plane. Here the persistance threshold given to DisPerSE is varied between no cuts (a), $\sigma=1$ (b), $\sigma=2$ (c), $\sigma=3$ (d). Each blue dot represents a halo particle while all lines represents filaments assigned by DisPerSE.}
\end{figure}
Figures \ref{fig:scatterMD1} and \ref{fig:histMD1} shows scatter plots and
histograms of filament lengths for the Multidark 1 dataset containing $534559$
halo particles.
\begin{figure}[htbp]
    \subfigure[]{\includegraphics[width=0.5\textwidth]{figures/scatterplots/scatter_MD2_all.pdf}}\hspace{1em}%
    \subfigure[]{\includegraphics[width=0.5\textwidth]{figures/scatterplots/scatter_MD2_s1.pdf}}
    \subfigure[]{\includegraphics[width=0.5\textwidth]{figures/scatterplots/scatter_MD2_s2.pdf}}\hspace{1em}%
    \subfigure[]{\includegraphics[width=0.5\textwidth]{figures/scatterplots/scatter_MD2_s3.pdf}}
    \caption{Figure showing a slice of the multidark 2 dataset for $975$Mpc/h $\leq z\leq1025$Mpc/h in the $x$, $y$ plane. Here the persistance threshold given to DisPerSE is varied between no cuts (a), $\sigma=1$ (b), $\sigma=2$ (c), $\sigma=3$ (d). Each blue dot represents a halo particle while all lines represents filaments assigned by DisPerSE.}
\end{figure}
\begin{figure}[htbp]
    \subfigure[]{\includegraphics[width=0.5\textwidth]{figures/histograms/filament_histMD2_all.pdf}}\hspace{1em}%
    \subfigure[]{\includegraphics[width=0.5\textwidth]{figures/histograms/filament_histMD2_s1.pdf}}
    \subfigure[]{\includegraphics[width=0.5\textwidth]{figures/histograms/filament_histMD2_s2.pdf}}\hspace{1em}%
    \subfigure[]{\includegraphics[width=0.5\textwidth]{figures/histograms/filament_histMD2_s3.pdf}}
    \caption{Figure showing histograms of filament lengths calculated by disperse the multidark 2 dataset for $975$Mpc/h $\leq z\leq1025$Mpc/h in the $x$, $y$ plane. Here the persistance threshold given to DisPerSE is varied between no cuts (a), $\sigma=1$ (b), $\sigma=2$ (c), $\sigma=3$ (d).}
\end{figure}
\subsection{Density profiles}
\subsection{Radial velocity profiles}
\subsection{Correlation function filament-galaxy $\xi_{fg}$}

\section{Void analysis}
\subsection{Histograms}
\begin{figure}[htbp]\label{fig:histMD2}
    \includegraphics{figures/histograms/void_histogramMD2.pdf}
    \caption{Figure showing a histogram of the effective void radius of voids found in the Multidark 2 dataset.}
\end{figure}
\begin{figure}[htbp]\label{fig:histMD3}
    \includegraphics{figures/histograms/void_histogramMD3.pdf}
    \caption{Figure showing a histogram of the effective void radius of voids found in the Multidark 2 dataset.}
\end{figure}
\begin{figure}[htbp]\label{fig:histMD4}
    \includegraphics{figures/histograms/void_histogramMD4.pdf}
    \caption{Figure showing a histogram of the effective void radius of voids found in the Multidark 2 dataset.}
\end{figure}
\subsection{Density profiles}

\begin{figure}[htbp]\label{fig:deltaMD2}
    \includegraphics{figures/Density_profiles/Voids/density_profileMD2.pdf}
    \caption{Figure showing the density profile for voids found in the Multidark 2 dataset.}
\end{figure}
\begin{figure}[htbp]\label{fig:deltaMD3}
    \includegraphics[width=1\textwidth]{figures/Density_profiles/Voids/density_profileMD3.pdf}
    \caption{Figure showing the density profile for voids found in the Multidark 3 dataset. Here the radius of voids considered are in the radius range of: all voids: top figure, $20$Mpc/h $\leq r\leq 40$Mpc/h in the bottom left, and $r\geq 40$Mpc/h in the bottom right.}
\end{figure}
\begin{figure}[htbp]\label{fig:deltaMD4}
    \includegraphics[width=1\textwidth]{figures/Density_profiles/Voids/density_profileMD4.pdf}
    \caption{Figure showing the density profile for voids found in the Multidark 4 dataset. Here the radius of voids considered are in the radius range of: all voids: top figure, $20$Mpc/h $\leq r\leq 40$Mpc/h in the bottom left, and $r\geq 40$Mpc/h in the bottom right.}
\end{figure}

\begin{figure}[htbp]\label{fig:DeltaMD2}
    \includegraphics{figures/Density_profiles/Voids/DeltaMD2.pdf}
    \caption{Figure showing the density contrast $\Delta_v(r)$ for voids found in the Multidark 2 dataset.}
\end{figure}
\begin{figure}[htbp]\label{fig:DeltaMD3}
    \includegraphics[width=1\textwidth]{figures/Density_profiles/Voids/DeltaMD3.pdf}
    \caption{Figure showing the density contrast $\Delta_v(r)$ for voids found in the Multidark 3 dataset. Here the radius of voids considered are in the radius range of: all voids: top figure, $20$Mpc/h $\leq r\leq 40$Mpc/h in the bottom left, and $r\geq 40$Mpc/h in the bottom right.}
\end{figure}
\begin{figure}[htbp]\label{fig:DeltaMD4}
    \includegraphics[width=1\textwidth]{figures/Density_profiles/Voids/DeltaMD4.pdf}
    \caption{Figure showing the density contrast $\Delta_v(r)$ for voids found in the Multidark 4 dataset. Here the radius of voids considered are in the radius range of: all voids: top figure, $20$Mpc/h $\leq r\leq 40$Mpc/h in the bottom left, and $r\geq 40$Mpc/h in the bottom right.}
\end{figure}
\subsection{Radial velocity profiles}
\begin{figure}[htbp]\label{fig:vrMD2}
    \includegraphics[width=1\textwidth]{figures/Theory_vs_v_r/Voids/vr_comparisonMD2.pdf}
    \caption{Figure showing the radial velocity profile halos around voids found in the Multidark 2 dataset. Here it is compared with the prediction from linear theory given in equation \ref{eq:vrvoid}. Here the radius of voids considered are in the radius range of: all voids: top figure, $20$Mpc/h $\leq r\leq 40$Mpc/h in the bottom left, and $r\geq 40$Mpc/h in the bottom right.}
\end{figure}

\begin{figure}[htbp]\label{fig:vrMD3}
    \includegraphics[width=1\textwidth]{figures/Theory_vs_v_r/Voids/vr_comparisonMD3.pdf}
    \caption{Figure showing the radial velocity profile halos around voids found in the Multidark 3 dataset. Here it is compared with the prediction from linear theory given in equation \ref{eq:vrvoid}. Here the radius of voids considered are in the radius range of: all voids: top figure, $20$Mpc/h $\leq r\leq 40$Mpc/h in the bottom left, and $r\geq 40$Mpc/h in the bottom right.}
\end{figure}

\begin{figure}[htbp]\label{fig:vrMD4}
    \includegraphics[width=1\textwidth]{figures/Theory_vs_v_r/Voids/vr_comparisonMD4.pdf}
    \caption{Figure showing the radial velocity profile halos around voids found in the Multidark 4 dataset. Here it is compared with the prediction from linear theory given in equation \ref{eq:vrvoid}. Here the radius of voids considered are in the radius range of: all voids: top figure, $20$Mpc/h $\leq r\leq 40$Mpc/h in the bottom left, and $r\geq 40$Mpc/h in the bottom right.}
\end{figure}

\begin{figure}[htbp]\label{fig:sigmavMD2}
    \includegraphics[width=1\textwidth]{figures/Theory_vs_v_r/Voids/sigma_vzMD2.pdf}
    \caption{Figure showing the velocity dispersion for halos around voids given by equation \ref{eq:sigma_v} found in the Multidark 2 dataset. Here the radius of voids considered are in the radius range of: all voids: top figure, $20$Mpc/h $\leq r\leq 40$Mpc/h in the bottom left, and $r\geq 40$Mpc/h in the bottom right.}
\end{figure}

\begin{figure}[htbp]\label{fig:sigmavMD3}
    \includegraphics[width=1\textwidth]{figures/Theory_vs_v_r/Voids/sigma_vzMD3.pdf}
    \caption{Figure showing the velocity dispersion for halos around voids given by equation \ref{eq:sigma_v} found in the Multidark 3 dataset. Here the radius of voids considered are in the radius range of: all voids: top figure, $20$Mpc/h $\leq r\leq 40$Mpc/h in the bottom left, and $r\geq 40$Mpc/h in the bottom right.}
\end{figure}

\begin{figure}[htbp]\label{fig:sigmavMD4}
    \includegraphics[width=1\textwidth]{figures/Theory_vs_v_r/Voids/sigma_vzMD4.pdf}
    \caption{Figure showing the velocity dispersion for halos around voids given by equation \ref{eq:sigma_v} found in the Multidark 4 dataset. H Here the radius of voids considered are in the radius range of: all voids: top figure, $20$Mpc/h $\leq r\leq 40$Mpc/h in the bottom left, and $r\geq 40$Mpc/h in the bottom right.}
\end{figure}
\section{Parameter fits}
\subsection{Linear bias approximation}

\begin{table}\label{tab:MD_linbias}
    \centering
    \footnotesize
    \begin{tabular}{| c | c | c | c | c | c |}
        \hline
        Dataset& $\epsilon$ & $\beta$ & $\sigma_v$  \\
        \hline
        MD2& $0.99749\pm 0.00912$ & $0.30204\pm 0.03110$ & $391.358\pm 33.368$\\ 
        MD3 no cuts. & $0.99999\pm 0.0.00891$ & $0.31852\pm 0.03236$ & $362.668\pm 27.289$\\
        MD3 $20$Mpc/h$\leq r\leq 40$ Mpc/h & $0.99905\pm 0.00753$ & $0.338485\pm 0.02905$ & $365.180\pm 22.197$\\
        MD3 $r\geq 40$Mpc/h & $0.99694\pm 0.01015$ & $0.37742\pm 0.03343$ & $312.802\pm 48.616$\\
        MD4 & $1.00351\pm 0.00876$ &  $0.30180\pm 0.03349$ & $322.655\pm 23.727$\\
        MD4 $20$Mpc/h$\leq r\leq 40$ Mpc/h & $1.0015\pm 0.00775$ & $0.35937\pm 0.03105$ & $315.542\pm 23.977$\\
        MD4 $r\geq 40$ Mpc/h & $0.99722\pm 0.01041$ & $0.41261\pm 0.03368$ & $255.907\pm 55.088$ \\
        \hline
    \end{tabular}
    \caption{Best fit values for the parameters $\epsilon$, $\beta$ and $\sigma_v$ for the different datasets using a linear bias approximation.}
\end{table}
\begin{figure}[htbp]\label{fig:linbiasMD2}
    \includegraphics[width=1\textwidth]{figures/cornerplots/linbias/MD2.pdf}
    \caption{Figure showing showing parameters fits to $\beta$, $\sigma_v$ and $\epsilon$ for the Multidark 2 dataset using a linear bias approximation and  modelling the velocity using linear theory given by equation \ref{eq:vrvoid}. Overplotted are the fiducial values for each parameter.}
\end{figure}

\begin{figure}[htbp]\label{fig:linbiasMD3}
    \includegraphics[width=1\textwidth]{figures/cornerplots/linbias/MD3.pdf}
    \caption{Figure showing showing parameters fits to $\beta$, $\sigma_v$ and $\epsilon$ for the Multidark 3 dataset using a linear bias approximation and  modelling the velocity using linear theory given by equation \ref{eq:vrvoid}. Overplotted are the fiducial values for each parameter.}
\end{figure}

\begin{figure}[htbp]\label{fig:linbiasMD3R2040}
    \includegraphics[width=1\textwidth]{figures/cornerplots/linbias/MD3_R20_40.pdf}
    \caption{Figure showing showing parameters fits to $\beta$, $\sigma_v$ and $\epsilon$ for the Multidark 3 dataset using a linear bias approximation, modelling the velocity using linear theory given by equation \ref{eq:vrvoid} and only considering voids with effective radius $20$Mpc/h$\leq r \leq 40$Mpc/h. Overplotted are the fiducial values for each parameter}
\end{figure}

\begin{figure}[htbp]\label{fig:linbiasMD3R40}
    \includegraphics[width=1\textwidth]{figures/cornerplots/linbias/MD3_R40_200.pdf}
    \caption{Figure showing showing parameters fits to $\beta$, $\sigma_v$ and $\epsilon$ for the Multidark 3 dataset using a linear bias approximation, modelling the velocity using linear theory given by equation \ref{eq:vrvoid} and only considering voids with effective radius $r\geq 40$Mpc/h. Overplotted are the fiducial values}
\end{figure}

\begin{figure}[htbp]\label{fig:linbiasMD4}
    \includegraphics[width=1\textwidth]{figures/cornerplots/linbias/MD4.pdf}
    \caption{Figure showing showing parameters fits to $\beta$, $\sigma_v$ and $\epsilon$ for the Multidark 4 dataset using a linear bias approximation and  modelling the velocity using linear theory given by equation \ref{eq:vrvoid}.
    Overplotted are the fiducial values for each parameter}
\end{figure}

\begin{figure}[htbp]\label{fig:linbiasMD4R2040}
    \includegraphics[width=1\textwidth]{figures/cornerplots/linbias/MD4_R20_40.pdf}
    \caption{Figure showing showing parameters fits to $\beta$, $\sigma_v$ and $\epsilon$ for the Multidark 4 dataset using a linear bias approximation, modelling the velocity using linear theory given by equation \ref{eq:vrvoid} and only consideringy voids with effective radius $20$Mpc/h$\leq r \leq 40$Mpc/h. Overplotted are the fiducial values for each parameter}
\end{figure}

\begin{figure}[htbp]\label{fig:linbiasMD4R40}
    \includegraphics[width=1\textwidth]{figures/cornerplots/linbias/MD4_R40_200.pdf}
    \caption{Figure showing showing parameters fits to $\beta$, $\sigma_v$ and $\epsilon$ for the Multidark 4 dataset using a linear bias approximation, modelling the velocity using linear theory given by equation \ref{eq:vrvoid} and only considering voids with effective radius $r\geq 40$Mpc/h. Overplotted are the fiducial values for each parameter.}
\end{figure}
\subsection{Linear bias approximation with Achitouv (2017) term}
\begin{figure}[htbp]\label{fig:linbiasMD2mod}
    \includegraphics[width=1\textwidth]{figures/cornerplots/v_correction/MD2_mod.pdf}
    \caption{Figure showing showing parameters fits to $\beta$, $\sigma_v$ and $\epsilon$ for the Multidark 2 dataset using a linear bias approximation, modelling the the proposed velocity correction by \cite{Achitouv_streaming} linear theory given by equation \ref{eq:achitouv2017}.}
\end{figure}

\begin{figure}[htbp]\label{fig:linbiasMD3mod}
    \includegraphics[width=1\textwidth]{figures/cornerplots/v_correction/MD3_mod.pdf}
    \caption{Figure showing showing parameters fits to $\beta$, $\sigma_v$ and $\epsilon$ for the Multidark 3 dataset using a linear bias approximation, modelling the the proposed velocity correction by \cite{Achitouv_streaming} linear theory given by equation \ref{eq:achitouv2017}.}
\end{figure}

\begin{figure}[htbp]\label{fig:linbiasMD3modR2040}
    \includegraphics[width=1\textwidth]{figures/cornerplots/v_correction/MD3_R20_40_mod.pdf}
    \caption{Figure showing showing parameters fits to $\beta$, $\sigma_v$ and $\epsilon$ for the Multidark 3 dataset using a linear bias approximation, modelling the the proposed velocity correction by \cite{Achitouv_streaming} linear theory given by equation \ref{eq:achitouv2017}. In this figure only voids with effective radius $20$Mpc/h$\leq r \leq 40$Mpc/h are considered.}
\end{figure}

\begin{figure}[htbp]\label{fig:linbiasMD3modR40}
    \includegraphics[width=1\textwidth]{figures/cornerplots/v_correction/MD3_R40_200_mod.pdf}
    \caption{Figure showing showing parameters fits to $\beta$, $\sigma_v$ and $\epsilon$ for the Multidark 3 dataset using a linear bias approximation, modelling the the proposed velocity correction by \cite{Achitouv_streaming} linear theory given by equation \ref{eq:achitouv2017}. In this figure only voids with effective radius $r \geq 40$Mpc/h are considered.}
\end{figure}


\begin{figure}[htbp]\label{fig:linbiasMD4mod}
    \includegraphics[width=1\textwidth]{figures/cornerplots/v_correction/MD4_mod.pdf}
    \caption{Figure showing showing parameters fits to $\beta$, $\sigma_v$ and $\epsilon$ for the Multidark 4 dataset using a linear bias approximation, modelling the the proposed velocity correction by \cite{Achitouv_streaming} linear theory given by equation \ref{eq:achitouv2017}.}
\end{figure}

\begin{figure}[htbp]\label{fig:linbiasMD4modR2040}
    \includegraphics[width=1\textwidth]{figures/cornerplots/v_correction/MD4_R20_40_mod.pdf}
    \caption{Figure showing showing parameters fits to $\beta$, $\sigma_v$ and $\epsilon$ for the Multidark 4 dataset using a linear bias approximation, modelling the the proposed velocity correction by \cite{Achitouv_streaming} linear theory given by equation \ref{eq:achitouv2017}. In this figure only voids with effective radius $20$Mpc/h$\leq r \leq 40$Mpc/h are considered.}
\end{figure}

\begin{figure}[htbp]\label{fig:linbiasMD4modR40}
    \includegraphics[width=1\textwidth]{figures/cornerplots/v_correction/MD4_R40_200_mod.pdf}
    \caption{Figure showing showing parameters fits to $\beta$, $\sigma_v$ and $\epsilon$ for the Multidark 4 dataset using a linear bias approximation, modelling the the proposed velocity correction by \cite{Achitouv_streaming} linear theory given by equation \ref{eq:achitouv2017}. In this figure only voids with effective radius $r \geq 40$Mpc/h are considered.}
\end{figure}

\subsection{Dark matter profile fits}

\begin{figure}[htbp]\label{fig:linbiasMD3DM}
    \includegraphics[width=1\textwidth]{figures/cornerplots/rscale/MD3_all_DM_rscale.pdf}
    \caption{Figure showing showbing parameters fits to $f\sigma_8$, $\sigma_v$ $r_{\mathrm{scale}}$ and $\epsilon$ for the Multidark 3 dataset using a dark matter profile as described in section \ref{sec:dm_calibrate} and modelling the velocity using linear theory as in equation \ref{eq:vrvoid}}
\end{figure}

\begin{figure}[htbp]\label{fig:linbiasMD3DMR40}
    \includegraphics[width=1\textwidth]{figures/cornerplots/rscale/MD3_R40_200_DM_rscale.pdf}
    \caption{Figure showing showing parameters fits to $f\sigma_8$, $\sigma_v$ $r_{\mathrm{scale}}$ and $\epsilon$ for the Multidark 3 dataset using a dark matter profile as described in section \ref{sec:dm_calibrate} and modelling the velocity using linear theory as in equation \ref{eq:vrvoid}. In this figure only voids with effective radius $r \geq 40$Mpc/h are considered.}
\end{figure}

\begin{figure}[htbp]\label{fig:linbiasMD4DM}
    \includegraphics[width=1\textwidth]{figures/cornerplots/rscale/MD4_all_DM_rscale.pdf}
    \caption{Figure showing showing parameters fits to $f\sigma_8$, $\sigma_v$ $r_{\mathrm{scale}}$ and $\epsilon$ for the Multidark 4 dataset using a dark matter profile as described in section \ref{sec:dm_calibrate} and modelling the velocity using linear theory as in equation \ref{eq:vrvoid}}
\end{figure}

\begin{figure}[htbp]\label{fig:linbiasMD4DMR40}
    \includegraphics[width=1\textwidth]{figures/cornerplots/rscale/MD4_R40_200_DM_rscale.pdf}
    \caption{Figure showing showing parameters fits to $f\sigma_8$, $\sigma_v$ $r_{\mathrm{scale}}$ and $\epsilon$ for the Multidark 4 dataset using a dark matter profile as described in section \ref{sec:dm_calibrate} and modelling the velocity using linear theory as in equation \ref{eq:vrvoid}. In this figure only voids with effective radius $r \geq 40$Mpc/h are considered.}
\end{figure}

\begin{table}\label{tab:MD_DM}
    \centering
    \footnotesize
    \begin{tabular}{| c | c | c | c | c | c |}
        \hline
        Dataset& $\epsilon$ & $f\sigma_8$ & $\sigma_v$ & $r_\mathrm{scale}$ \\
        \hline
        MD2& $0.99947\pm 0.00988$ & $0.53545\pm 0.60226$ & $352.832\pm 34.967$ & $0.94749\pm 0.05621$\\ 
        MD3 no cuts. & $1.00110\pm 0.00950$ & $0.35727\pm 0.0.40479$ & $293.070\pm 29.288$ & $0.91153\pm 0.59305$ \\
        MD3 $r\geq 40$Mpc/h & $0.99906\pm 0.01072$ & $0.65484\pm 0.06126$ & $283.288\pm 52.371$ & $1.03431\pm 0.05075$\\
        MD4 & $0.99823\pm 0.00935$ &  $0.2340\pm 0.02941$ & $232.614\pm 25.018$ & $0.91541\pm 0.07299$\\
        MD4 $r\geq 40$ Mpc/h & $0.99543\pm 0.01123$ & $0.60042\pm 0.04589$ & $154.538\pm 45.498$ & $1.03838\pm 0.04681$ \\
        \hline
    \end{tabular}
    \caption{Best fit values for the parameters $\epsilon$, $\sigma_v$, $f \sigma_8$ and $r_\mathrm{scale}$ for the different datasets using a dark matter density profile.}
\end{table}

\subsection{Correlation function void-galaxy $\xi_{vg}$}


