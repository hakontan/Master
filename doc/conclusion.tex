\chapter{Concluding remarks}\label{sec:conclusion}
In this thesis I have derived and tested a simple model for modelling the velocity of matter around cosmic filaments using linear theory and compared it to data from numerical simulations. I have also tested a proposed improvement to the radial velocity of matter around voids and applied it to a model of the void-galaxy cross correlation function using a linear bias approximation. This was compared to the regular model for the radial velocity derived from linear theory. For these velocity models the quadrupole of the void-galaxy cross correlation was compared by applying cuts to the dataset including voids of different sizes in order to see if this affected the validity of the velocity approximations. Lastly, using a dark matter overdensity profile from another dataset, the regular velocity model was tested to see if potential corrections could be applied to an arbitrary density profile while still providing accurate parameter fits.
\section{Summary of main results}\label{sec:mainresults}
Using linear theory a simple model, similar to the model already tested on voids, the radial velocity perpendicular to the spine of cosmological filaments was tested. Overall this model looked promising, however, the comparison with numerical data illustrated the presence of irregularities regarding the amplitude of the velocity profile close to the filament spine. The amplitude of the stacked radial velocity profile for filaments showed large deviations depending on the dataset and $\sigma$-threshold chosen as input to the DisPerSE code. The promising aspect of the model is that it accurately replicated the shape of the numerically calculated velocity profile. The modelling of the radial velocity profile for filaments is reliant on accurate modelling of the density profile. The density profile for the filaments showed varying densities close to the filament spine depending on dataset and $\sigma$-threshold chosen. This may be an effect of how DisPerSE applies cuts to the filament catalogue which in turn may affect the volume scaling when calculating the number density in each cylinder shell. It may also be a reasonable result and in turn will suggest that linear theory is not a reasonable approximation to radial velocity of matter around filaments. However, on the basis of the results presented in this thesis, one is not able to draw a conclusion of wether or not linear theory is a sufficient approximation for the behavior of matter around filaments.\\\indent
In this thesis the regular linear model for the radial velocity of matter around filaments was also tested. In addition, a proposed improvement to the regular velocity model was tested and both models were compared with numerical data. Both of these velocity models was implemented into the anisotropic void-galaxy cross correlation function and their performance was tested in terms of fitting cosmological parameters. When conducting the comparison between the two parameters different cuts to the dataset was applied. The $\epsilon$ parameter, in which is directly linked to the cosmological parameters, recieved consistently accurate fits using the regular velocity model. The model with the velocity correction term was biased in the sense that it consistently underestimated fits to $\epsilon$. On the other hand, the model with the correction term made better fits to the $\beta$ parameter when small voids was considered. However, when only large voids were considered the regular model consistently provided the best fits for both of these parameter for the two largest datasets and for the MD2 dataset. The modelling of the velocity itself showed that when applying cuts only keeping the larger voids, the regular model qualitatively resembled the numerically calculated radial velocity in an accurate fashion. When applying cuts including only smaller voids there was not as prominent of an allround difference but both models seemed to capture certain features better. The regular model captured better the velocity on large scales while the amplitude predicted by the model with correction was more accurate with these cuts. These results suggest that when applying cuts to the dataset, excluding small voids, in which may be subject to non-linear behavior, the regular velocity model performed better.\\\indent
Finally, the model for the anisotropic cross-correlation function between voids and galaxies was tested using a dark matter overdensity profile gathered from another dataset. Using this overdensity profile only the regular velocity model was implemented. For this test the cuts only considering smaller voids was excluded. Across all datasets and cuts this model managed to provide accurate samples for the $\epsilon$ parameter. Using no cuts, the model provided unsatisfying samples for $f\sigma_8$. This improved however when cuts only considering larger voids was considered, but the sampling was still not accurate enough for this implementation of the model to be deemed sufficient.  This suggests that the model was not able to sample the amplitude of the cross-correlation function accurately. The scaling of the overdensity profile, parametrized by the $r_{scale}$ paramater, did however show that the model captured in which direction the density profile should be shifted. Overall this analysis suggests that when sampling the $\epsilon$ parameter using an arbitrary dark matter overdensity profile and correcting it may be sufficient. However, if one wants to sample $f\sigma_8$, the methodology applied in this analysis may not be sufficient.
\section{Suggestions for future work}\label{sec:futurework}