\chapter{Introduction}
The current dominating model for the universe is the $\Lambda CDM$ model. This model assumes that the mass-energy density of the universe is made up of three major components, namely dark energy $\Lambda$, cold dark matter CDM and regular baryonic matter. The modeling of the evolution and expansion of the universe is related to the mass-energy density of the universe itself. An important aspect of cosmology is therefore to constrain the bounds on these components in order to figure out how large the fraction of the total mass-energy density of the universe each of them makes up.\\\indent
The leading experiments related to constraining cosmological parameters is the study of radiation remnant from the very early universe, namely the cosmological microwave background (CMB)\cite{1965cmb}, with the COBE-\cite{Smoot_1999} WMAP-\cite{Wmap} and Planck-satellites\cite{planckvi} in the forefront. Other studies include that of stellar supernova type-Ia explosions\cite{Pantheon}. Advances in cosmological experiments over the last decades have brought cosmology into the era of "precision cosmology", now that cosmological parameters can be constrained with sub-percent accuracy. This has lead to tensions in that certain experiments may disagree over values for cosmological parameters. An example of this is the Hubble tension in which measurements of the CMB disagree with that of type Ia supernovae when measuring the evolution of the Hubble parameter, which quantifies the expansion of the universe. This particular tension could be a case of statistical fluctuation or systematic error, but these situations do warrant more independent experiments in order to strengthen the current knowledge.\\\indent Additional independent sets of experiments relating cosmological components to observables can be found through studies of the distribution of matter in the universe, namely the large-scale structure (LSS) of the universe. Studies of the large-scale structure of the universe include the study of how matter is clustered at different cosmological redshifts in order to determine statistical properties of the matter distribution. On large scales, the matter distribution in the universe forms a web-like pattern known as the cosmic web\cite{bondweb}. Substructures of the cosmic web include voids, empty regions of space containing little to no matter. These voids are bound by structures known as filaments. Filaments consist of gravitationally bound galaxy clusters. \\\indent
One of the main statistical properties of the LSS that one can measure is the angular correlation function between definable physical objects. In short, a correlation function in this context measures the excess probability over random that two objects are separated by a distance $r$. The imprints of baryonic acoustic oscillations on the correlation between galaxy pairs\cite{pe00300h}, which appear as ripples in the density distribution of baryonic matter, serve as a standard ruler when measuring cosmological parameters. The effects this has on the correlation function between matter has been studied both theoretically\cite{peebles1980} and through observational experiments\cite{Eisenstein_2005}. In addition, other studies like \cite{refId0} and \cite{Bautista_2020} show that the statistical properties of galaxy clustering provides useful constraints on cosmological parameters. Another statistical tool that allows for the extraction of information from the density field is the void-galaxy cross-correlation function. The fact that the universe is expanding\cite{Hubble168}, causes light travelling towards us to be redshifted. With current observational experiments, the only reliable way to observe the LSS is through observations of electromagnetic radiation. When observing objects in the universe, the redshifted light causes the observed matter distribution to appear distorted along the line of sight direction. This introduces anisotropy in the void-galaxy correlation function. Work on modelling the anisotropic effects on the void-galaxy correlation function as a result of redshift space distortions has been conducted \cite{Nadathur_corr}. Additional works like \cite{BeyondBAO} and \cite{Nadathur_2020} have shown that this provides additional constraints to cosmological parameters. These independent experiments allow one to extract even more information about underlying cosmology from the density field.\\\indent
The filaments themselves have not yet received the proper treatment as that of voids and galaxy clusters. Although great efforts have been made analysing many of their properties\cite{Libeskind_2017}, their matter distribution has not yet received a thorough statistical review as that of voids and the matter distribution in general. Further contribution to this field includes proper modelling of the physical behavior, such as the velocity field, of matter around filaments. By modelling the statistical properties of filaments correctly, one can potentially gain an additional set of experiments for extracting cosmological properties from the density field. This can potentially serve as a useful supplement for the studies conducted on voids and galaxy clustering. Additionally, future galaxy surveys are expected to greatly increase the size of observational datasets. Current surveys of the large scale structure of the universe include the Sloan Digital Sky Survey \cite{Eisenstein_2011}, in which the redshifts of $1.5$ million galaxies is measured. Future surveys include the Euclid mission, in which one of the scientific goals is to measure the redshifts of upwards of $50$ million galaxies\cite{eucliddefinition}. In preparation for future experiments like Euclid, which will greatly improve the data for which LSS studies are to be carried out, simulation data is useful for testing and developing methodology. Current computer simulations enable the study of the evolution of billions of dark matter particles\cite{Millennium}\cite{Multidark_dataset}, which allows for relatively accurate representations of the real universe. Since future cosmological surveys and experiments allow for even more precise measurements, additional emphasis on mapping the potential error of applied models is crucial. This is in order to fully extract as much knowledge as possible from future surveys and experiments. As mentioned, the correlation function for voids and galaxies has been studied in previous works. However, in preparation for future observational experiments, applying methodology and potential improvements on large datasets provide useful testing.\\\indent
The work in this thesis is divided into two main parts. Using data from The Big MultiDark Planck simulation (BigMDPL)\cite{Multidark_dataset}, the first part of this work contains an analysis of filaments where the goal is to test a model for the velocity component of dark matter halos perpendicular to the filament spine. Using linear perturbation theory, I will derive a simple model for the radial velocity perpendicular to the spine of cosmological filaments. This model will be compared to velocity data from numerical simulations in order to test whether this is a valid approximation for it to be used in further modelling of the statistical properties of filaments.
The second part of this thesis includes a test of the void-galaxy correlation function applied on data from The Big MultiDark Planck simulation (BigMDPL)\cite{Multidark_dataset}. The methodology will be tested using three different samples with a different number of dark matter halos ranging from $3.5$ million to $15$ million. In addition, proposed improvements to the modelling of velocities of matter around voids\cite{Achitouv_streaming} will be tested in order to see if this will improve the performance of the models.\\\indent
This thesis is divided into multiple sections. Chapter \ref{sec:backgroundtheory} provides a summary of cosmological theory used in the development of the methodology. It also provides a short summary of the fundamental statistical theory used and cosmological simulations in general. Chapter \ref{sec:codesused} provides an introduction to the codes utilized for identifying filaments and voids from the simulated distribution of dark matter halos. This is followed by a short introduction of the dataset and its properties given in chapter \ref{sec:dataset}. The methodology particular to the work conducted in this thesis itself is introduced in chapter \ref{sec:method}. This is followed chapter \ref{sec:results} which presents the results and a discussion of their properties. This section is split into two parts. Section \ref{sec:filaments} provides the results and a discussion for the work done on filaments, while section \ref{sec:void} provides the same for the work done on voids. Finally, some concluding remarks are given in chapter \ref{sec:conclusion}. 