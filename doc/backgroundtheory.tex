\chapter{Background Theory}
\section{History of cosmology and how we model the universe}

\section{Cosmological parameters and distances}
When describing an expanding universe
\subsection{Evolution of the universe and the $\Lambda$CDM model}
In modern cosmology, the dominating model for describing the universe is the $\Lambda$CDM model. Its name is an abbreviation for what 
is considered the main energy contributions governing the expansion of the universe. The cosmological constant $\Lambda$ represents dark energy, CDM is an abbrevation for cold dark matter
and lastly we have ordinary matter which is what we interact with in our everyday lives. The current estimates suggest that approximately $69\%$ of the universe consists of dark energy while the remaining $31\%$ (Dette er hentet fra Planck 2018) is attributed to dark matter at around 
$27\%$ and $4\%$ for regular matter leaving only trace contributions from other energy contributing factors such as photons ($\gamma$) and neutrinos ($\nu$).(Kanskje være mer spesifikk her). When modelling the universe these quantites enter into 
what is called density parameters $\Omega_i$, where $i$ represents a certain type of energy contribution to the universe i.e dark matter or CDM. The density parameter is defined as
\begin{equation}
    \Omega_i = \frac{\rho_i}{\rho_c},
\end{equation}
where $\rho_i$ is the density of the current energy contribution and $\rho_c$ is the critical density of the universe. The critical density is the density at which the (blabla). For a flat universe we have
\begin{equation}
    \sum_i \Omega_i = 1.
\end{equation}




