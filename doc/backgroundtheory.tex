\chapter{Background Theory}
\section{History of cosmology and how we model the universe}
\subsection{Distances in the universe and cosmological redshift}
Distances in an expanding universe can prove hard to fathom. In our everyday
lives we are used to measuring distances in meters and kilometers. The
average distance from the earth to the sun however is approximately $1.49\cdot10^{11}$m.
Allthough this distance on a cosmic scale is very short, the numbers in our
everyday units of measurement already start to become too big for us to have a
reasonable idea of how long this distance actually is. We have a clear idea of
what a meter and a kilometer is as we have to deal with them everyday. To make
things more manageable when measuring these large distances the average distance
from the earth to the sun has become its own distance unit known as the
Astronomical unit (AU). This unit is a great tool for measuring relative
distances in our own solar system, but our nearest star, Proxima Centauri, is
located at approximately $268394$AU away from our closest star the Sun.
Therefore it is convenient to introduce another unit of distance measurement known as light year. This unit with a value of $9.4607\cdot10^{15}$m is the distance at
which light travels in vacuum during one Julian year equal to $365.25$ days.
This gives us a more manageable account of the distance from the Sun to Proxima
Centauri as $4.244$ light years. With both the AU and the light year introduced
we can define the last unit of measuring distance, namely the parsec equal to $3.26$
light years. One parsec is equal to the distance at which $1$ AU subtends an angle
of one arcsecond (setningen er lik paa wikipedia og Barbara ryden). Most
cosmological distances are measured in parsec, abbreviated Pc, with either the
prefix mega or giga as cosmology deals mostly with intergalactic scales much
larger than the average distance between two galaxies.\\

Due to the fact that the universe is expanding, light traveling towards us from
far away objects gets redshifted. Since the expansion of the universe is
isotropic, i.e, the expansion rate of the universe is equal in all directions,
the redshift of this light can be used to measure the distance at which we
observe objects in the universe. The cosmic redshift $z$ is given by the
relation
\begin{equation}
    z = \frac{\lambda_{obs}}{\lambda_{em}} - 1,
\end{equation}
where $\lambda_{obs}$ is the observed wavelength we measure here on earth and
$\lambda_{em}$ is the wavelength emitted by the observed object. by combining this with the doppler formula
$\frac{\lambda_{obs}-\lambda_{em}}{\lambda_{em}} = \frac{v}{c}$, one gets
\begin{equation}
    z = \frac{v}{c},
\end{equation}
where $v$ is the velocity of the observed object relative to the observer. The
velocity of an object due to the expansion of the universe is given by Hubbles law
\begin{equation}
    v = H_0 d_p,
\end{equation}
where $d_p$ is what is known as proper distance and $H_0$ is the Hubble
parameter at present time. The Hubble parameter measures the expansion rate of
the universe as a function of time. (Kanskje prate om verdien av H0). Proper distance can be interpreted as the length measured if one
was to use a ruler between two objects $a$ and $b$ at a fixed time $t$ taking
into account the fact that our universe is expanding. Before introducing how proper distance is calculated it can be
useful to talk about the concept of comoving distance first. If we choose a time
$t=t_0$, which is present time, and introduce a coordinate system at this time,
we can choose an arbitrary point to center ourselves in in this coordinate
system. If we apply a ruler in this coordinate system and measure the distance
from our point to another arbitrary point, this is the comoving distance $r$.
The comoving distance remains constant through time as the universe expands. The
comoving distance is simply the distance measured in a coordinate system at
$t=t_0$ (dette er kanskje en krokete formulering $d_c$). The proper distance on the other hand factors in the expansion of the
universe through the scale factor $a(t)$. The proper distance is given as
\begin{equation}
    d_p = \int_{r_0}^{r_1}a(t)rdr,
\end{equation}
Where $a(t)$ is a quantity known as the scale factor. The scale factor parametrises the relative expansion of the
universe. It is defined in such a way that $a_0$, subscript $0$ meaning
$a(t=t_0)$ where $t_0$ is present time, is equal to one. This definition of
proper distance also implies that comoving distance is the proper distance at $t=t_0$.
With these properties in place we can now see how redshift, which is a
measurable quantity, can be used to measure the real distance to objects in our universe.\\

Another measurable measurable quantity is flux emitted from an astronomical object.
The flux $F$ is defined as
\begin{equation}
    F = \frac{L}{4\pi R^2},
\end{equation}
where $L$ is the luminosity and $R$ is the distance to the object. Luminosity is a measure
of the absolute electromagnetic power emitted from and object. The flux however, measured in $W/m^2$, is simply the luminosity
divided by the area of a shell with a radius at which we observe the object, assuming the radiation is isotropically emitted. Some astronomical phenomena, like a type 1a supernova, have a fixed luminosity
due to the characteristics of the process that creates it. This is called a standard candle. By measuring the observed flux from such a phenomena, one can then
calculate the the luminosity distance as 
\begin{equation}
    d_L = \sqrt{\frac{L}{4\pi F}}.
\end{equation}
(Kanskje si noe mer om svakheter med denne.)


\subsection{Modelling the evolution of the universe and the $\Lambda$CDM model}
The cosmological principle states that on sufficiently
large scales the universe is isotropic and homogenous. This means the universe
looks the same in every direction for every observer at any point in space for
sufficiently large scales. By sufficiently large scales one means
distances roughly $100$Mpc or larger (Ryden). This has been proved to be a good
assumption by surveys like the Sloan sky survey (svak formulering. Trenger ogsaa kilde). This was an important assumption
for a particular solution of the Einstein field equation. This equation, first published by
Albert Einstein in $1915$ in relation to his theory of general relativity. This equation
relates the geometry of spacetime to the distribution of matter within it.
The Friedman-Lemaître-Robertson-Walker metric, which assumes the universe is
homogenous, isotropic and expanding provides an analytical solution to the
Einstein equation (kanskje forklare hva en metrikk er for noe). Using this
metric together with the Einstein field equations results in the important
equations known as the first and second Friedman equations respectively as
\begin{equation}
    \frac{\dot{a}^2 + kc^2}{a^2} = \frac{8\pi G\rho + \Lambda c^2}{3}
\end{equation}
and
\begin{equation}
    \frac{\ddot{a}^2}{a^2} = -\frac{4\pi G}{3}(\rho + \frac{3p}{c^2}) + \frac{\Lambda c^2}{3}.
\end{equation}
(Kanskje inkludere denne utledningen fra kosmo 2 i et appendiks)
These differential equations relate evolution of the scale factor $a$ with the
following properties of the universe density $\rho$, pressure $p$, cosmological constant $\Lambda$, gravitational
constant $G$ and the speed of light $c$ (ikke glem k). For a flat universe $k=0$
(Hva er et flatt univers). In these equations $\dot{a}$ and $\ddot{a}$ dot denotes
derivative and double derivative with respect to time, meaning that $a=a(t)$ is a
function of time. \\

In modern cosmology, the dominating model for describing the universe is the $\Lambda$CDM model. Its name is an abbreviation for what 
is considered the main energy contributions governing the expansion of the universe. The cosmological constant $\Lambda$ represents dark energy, CDM is an abbrevation for cold dark matter
and lastly we have ordinary matter which is what we interact with in our everyday lives. The current estimates suggest that approximately $69\%$ of the universe consists of dark energy while the remaining $31\%$ (Dette er hentet fra Planck 2018) is attributed to dark matter at around 
$27\%$ and $4\%$ for regular matter leaving only trace contributions from other energy contributing factors such as photons ($\gamma$) and neutrinos ($\nu$).(Kanskje vaere mer spesifikk her) When modelling the universe these quantites enter into 
what is called density parameters $\Omega_i$, where $i$ represents a certain type of energy contribution to the universe i.e dark matter or CDM. The density parameter is defined as
\begin{equation}
    \Omega_{i,0} = \frac{\rho_{i,0}}{\rho_{c,0}},
\end{equation}
where $\rho_{i,0}i$ is the density of the current energy contribution and $\rho_{c,0}$ is
the critical density of the universe. Subscript zero indicating these quantities
measured at present time.The critical density is the density at
which gravity counteracts the expansion and expansion will eventually stop
(Dette er jeg veldig usikker på om er riktig å si). The critical density at a
given time is given as $\rho_c=\frac{3H(t)^2}{8\pi G}$. For a flat universe we have
\begin{equation}
    \sum_i \Omega_i = 1.
\end{equation}
A given energy density evolves through time with the scale
factor as
\begin{equation}
    \rho_i=\rho_{i,0}a^{-3(1+w_i)},
\end{equation}
where $w_i$ is a constant depending on what energy density contribution one is
considering. Summing over all energy
density contributions and using the fact that the Hubble parameter can be
expressed as $H=\frac{\dot{a}}{a}$, one can rewrite the first friedman equation
as
\begin{equation}
    H(t)=H_0\sqrt{(\Omega_{CDM,0} + \Omega_{b})a^{-3} + \Omega_{r,0}a^{-4} + \Omega_{\Lambda,0}},
\end{equation}
Where $b$ represents baryonic matter and $r$ is radiation
(Her trengs det kanskje mer forklaring og er kanskje ikke den riktige versjonen
av likningen.) 



