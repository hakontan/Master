\chapter{Introduction}
The current dominating model for the universe is the $\Lambda CDM$ model. This model assumes that the mass-energy density of the universe is made up of three major components, namely dark energy $\Lambda$, cold dark matter CDM and regular baryonic matter. The modelling of the evolution and expansion of the universe is related to the mass-energy density of the universe itself. An important aspect of cosmology is therefore to constrain the bounds on these components in order to figure out how large the fraction of the total mass-energy density of the universe each of them make up.\\\indent
The leading expirements related to constraining cosmological parameters is the study of radiation remnant from the very early universe, namely the cosmological microwave background (CMB)\cite{1965cmb}, with the WMAP\cite{Wmap} and Planck satellites\cite{planckvi} in the forefront. Other studies include that of stellar supernova type-Ia explosions\cite{Pantheon}. Advances in cosmological experiments over the last decades has brought cosmology into the era of "precision cosmology", now that cosmological parameters can be constrained with sub-percent accuracy. This has lead to tensions in that certain experiments may disagree over values for cosmological parameters. An example of this is the Hubble tension in which measurements of the CMB disagree with that of type Ia supernova when measuring the evolution of the Hubble parameter, in which quantifies the expansion of the universe. This particular tension could be a case of statistical fluctuation or systematic error, but these situations do warrant more independent experiments in order to strengthen the current knowledge.\\\indent Additional independent set of experiments relating cosmological components to observables can be found through studies of the distribution of matter in the universe, namely the large scale structure (LSS) of the universe. Studies of the large scale structure of the universe include the study of how matter is clustered at different cosmological redshifts in order to determine statistical properties of the matter distribution. On large scales the matter distribution in the universe form a web like pattern known as the cosmic web\cite{bondweb}. Substructures of the cosmic web include voids, empty regions of space containing little to no matter. These voids are bound by structures known as filaments. Filaments consists of gravitationally bound galaxy clusters. Studies like \cite{BeyondBAO}\cite{Nadathur_2020} have shown that the studies the statistical distribution of matter around voids is a usefull addition to the current expirements for studying properties of the universe. In addition other studies like \cite{refId0} show that the statistical properties of galaxy clustering provides usefull insight. \\\indent
The filaments themselves have not yet recieved the proper treatment as that of voids and galaxy clusters. Allthough great efforts have been made analysing their properties\cite{Libeskind_2017}, their matter distribution has not yet recieved a thorough statistical review as that of voids and the matter distribution in general. In order for this to happen proper modelling of the physical behaviour of matter around filaments has to be developed and tested. Using data from the The Big MultiDark Planck simulation (BigMDPL)\cite{Multidark_dataset}, the first part of this work contains an analysis of filaments where the goal is to test a model for the velocity component of dark matter halos perpendicual to the filament spine. Current surveys of the large scale structure of the universe include the Sloan Digital Sky Survey \cite{Eisenstein_2011} in which the redshifts of $1.5$ million galaxies is measured. Future surveys include the Euclid mission in which one of the scientific goals is to measure the redshifts of $50$ million galaxies\cite{eucliddefinition}.