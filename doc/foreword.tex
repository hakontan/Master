\chapter*{Abstract}
\addcontentsline{toc}{chapter}{\numberline{}Abstract}
The large scale structure of the universe provides independent sets of experiments for extracting information about cosmological parameters. Due to redshift space distortions, anisotropy is introduced when observing the matter distribution in the universe.
Future cosmological surveys, such as Euclid, is expected to greatly increase the number of galaxies having their redshifts measured thereby increasing the size of datasets. In preparation for conducting analysis on future datasets, methodology has to be thoroughly tested in order to extract as much information cosmological information from the density field as possible.\\\indent
In this thesis I derive and implement a model for the radial velocity of matter around cosmological filaments using linear perturbation theory. I will also implement and test a proposed improvement to the radial velocity of matter surrounding voids\cite{Achitouv_streaming}. This model, together with the regular radial velocity derived from linear theory, will be implemented into the void-galaxy cross correlation function in redshift space using a linear bias approximation for the overdensity profile. In addition, the regular radial velocity proposed by linear theory will also be implemented for into the void-galaxy cross correlation function in redshift space using an arbitrary dark matter overdensity profile. This dark matter profile will then be scaled in the radial direction introducing an $r_{scale}$ parameter and the amplitude is scaled by $f\sigma_8$. Using The Big MultiDark Planck simulation (BigMDPL)\cite{Multidark_dataset}, I will compare these models with data from numerical calculations and assess their performance. For the analysis conducted on voids, fits to the dataset was performed for the void-galaxy cross correlation function in redshift space. 
\\\indent
I find the model to provide promising results when predicting the radial velocity of matter around cosmological filaments. Although the amplitude of the velocity predicted by linear theory is not entirely accurate, it manages to replicate the shape of the radial velocity derived from the dataset. When applying fits to the void-galaxy correlation function in redshift space, I find the regular velocity model derived from linear theory to perform consistently better than the model with the proposed improvement when fitting $\epsilon$. Applying cuts to the dataset, including only voids with an effective radius $r>40$Mpc/h, i find the regular model to also perform better when fitting the $\beta$ parameter. Lastly, I find the parameter fits to the model using the dark matter overdensity profile to consistently provide accurate fits for $\epsilon$. However, the model does not provide accurate fits for the $f\sigma_8$ even with cuts applied.
\\\indent