\chapter{Background Theory}
\section{History of cosmology and how we model the universe}
\subsection{Distances in the universe}
Distances in an expanding universe can prove hard to fathom. In our everyday
lives we are used to measuring distances in meters and kilometers. The
average distance from the earth to the sun however is approximately $1.49\cdot10^{11}$m.
Allthough this distance on a cosmic scale is very short, the numbers in our
everyday units of measurement already start to become too big for us to have a
reasonable idea of how long this distance actually is. We have a clear idea of
what a meter and a kilometer is as we have to deal with them everyday. To make
things more manageable when measuring these large distances the average distance
from the earth to the sun has become its own distance unit known as the
Astronomical unit (AU). This unit is a great tool for measuring relative
distances in our own solar system, but our nearest star, Proxima Centauri, is
located at approximately $268394.804$AU away from our closest star the Sun.
Therefore another unit of distance measurement known as light year is
introduced. This unit with a value of $9.4607\cdot10^15$m is the distance in
which light travels in vacuum during one Julian year equal to $365.25$ days.
This gives us a more manageable account of the distance from the Sun to Proxima
Centauri as $4.244$ light years. With both the AU and the light year introduced
we can define the last unit of measuring distance, namely the parsec equal to $3.26$
light years. One parsec is equal to the distance at which $1$ AU subtends an angle
of one arcsecond (setningen er lik paa wikipedia og Barbara ryden). Most
cosmological distances are measured in parsec, abbreviated Pc, with either the
prefix mega or giga as cosmology deals mostly with intergalactic scales much
larger than the average distance between two galaxies.\\



\subsection{Modelling the evolution of the universe and the $\Lambda$CDM model}
The cosmological principle states that on sufficiently
large scales the universe is isotropic and homogenous. This means the universe
looks the same in every direction for every observer at any point in space for
sufficiently large scales. By sufficiently large scales one means
distances roughly $100$Mpc or larger (Ryden). This has been proved to be a good
assumption by surveys like the Sloan sky survey (svak formulering. Trenger ogsaa kilde). This was an important assumption
for a particular solution of the Einstein field equation. This equation, first published by
Albert Einstein in $1915$ in relation to his theory of general relativity,
relates the geometry of spacetime to the distribution of matter within it.
The Friedman-Lemaître-Robertson-Walker metric, which assumes the universe is
homogenous, isotropic and expanding provides an analytical solution to the
Einstein equation (kanskje forklare hva en metrikk er for noe). Using this
metric together with the Einstein field equations results in the important
equations known as the first and second Friedman equations respectively as
\begin{equation}
    \frac{\dot{a}^2 + kc^2}{a^2} = \frac{8\pi G\rho + \Lambda c^2}{3}
\end{equation}
and
\begin{equation}
    \frac{\ddot{a}^2}{a^2} = -\frac{4\pi G}{3}(\rho + \frac{3p}{c^2}) + \frac{\Lambda c^2}{3}.
\end{equation}
These differential equations relate evolution of the scale factor $a$ with the
following properties of the universe density $\rho$, pressure $p$, cosmological constant $\Lambda$, gravitational
constant $G$ and the speed of light $c$ (ikke glem k). For a flat universe $k=0$
(Hva er et flatt univers). In these equations $\dot{a}$ and $\ddot{a}$ dot denotes
derivative and double derivative with respect to meaning that $a=a(t)$ is a
function of time. The scale factor parametrises the relative expansion of the
universe. It is defined in such a way that $a_0$, subscript $0$ meaning
$a(t=t_0)$ where $t_0$ is present time, is equal to one.
\\
In modern cosmology, the dominating model for describing the universe is the $\Lambda$CDM model. Its name is an abbreviation for what 
is considered the main energy contributions governing the expansion of the universe. The cosmological constant $\Lambda$ represents dark energy, CDM is an abbrevation for cold dark matter
and lastly we have ordinary matter which is what we interact with in our everyday lives. The current estimates suggest that approximately $69\%$ of the universe consists of dark energy while the remaining $31\%$ (Dette er hentet fra Planck 2018) is attributed to dark matter at around 
$27\%$ and $4\%$ for regular matter leaving only trace contributions from other energy contributing factors such as photons ($\gamma$) and neutrinos ($\nu$).(Kanskje vaere mer spesifikk her) When modelling the universe these quantites enter into 
what is called density parameters $\Omega_i$, where $i$ represents a certain type of energy contribution to the universe i.e dark matter or CDM. The density parameter is defined as
\begin{equation}
    \Omega_i = \frac{\rho_i}{\rho_c},
\end{equation}
where $\rho_i$ is the density of the current energy contribution and $\rho_c$ is the critical density of the universe. The critical density is the density at which the (blabla). For a flat universe we have
\begin{equation}
    \sum_i \Omega_i = 1.
\end{equation}






