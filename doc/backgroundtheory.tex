\chapter{Background Theory}
\section{History of cosmology and how we model the universe}
Skrive noe generelle ting om kosmologi og astronomi historie her.
Følgende kapittel viser grunnleggende bakgrunnsteori som tesen bygger på.\\

The following subsections are largely based on \cite[ch.7]{schneider2006extragalactic}, \cite[ch.2]{Dodelson:1282338} and \cite{ryden2017introduction} unless otherwise noted.
\subsection{Distances in the universe and cosmological redshift}\label{sec:sec_distance}
Distances in an expanding universe can prove hard to fathom. In our everyday
lives we are used to measuring distances in meters and kilometers. The
average distance from the earth to the sun however is approximately $1.49\cdot10^{11}$m.
Allthough this distance on a cosmic scale is very short, the numbers in our
everyday units of measurement already start to become too big for us to have a
reasonable idea of how long this distance actually is. We have a clear idea of
what a meter and a kilometer is as we have to deal with them everyday. To make
things more manageable when measuring these large distances the average distance
from the earth to the sun has become its own distance unit known as the
Astronomical unit (AU). This unit is a great tool for measuring relative
distances in our own solar system, but our nearest star, Proxima Centauri, is
located at approximately $268394$AU away from our closest star the Sun.
Therefore it is convenient to introduce another unit of distance measurement known as light year. This unit with a value of $9.4607\cdot10^{15}$m is the distance at
which light travels in vacuum during one Julian year equal to $365.25$ days.
This gives us a more manageable account of the distance from the Sun to Proxima
Centauri as $4.244$ light years. With both the AU and the light year introduced
we can define the last unit of measuring distance, namely the parsec equal to $3.26$
light years. One parsec is equal to the distance at which $1$ AU subtends an angle
of one arcsecond (setningen er lik paa wikipedia og Barbara ryden). Most
cosmological distances are measured in parsec, abbreviated Pc, with either the
prefix mega or giga as cosmology deals mostly with intergalactic scales much
larger than the average distance between two galaxies.\\

Due to the fact that the universe is expanding, light traveling towards us from
far away objects gets redshifted. Since the expansion of the universe is
isotropic, i.e, the expansion rate of the universe is equal in all directions,
the redshift of this light can be used to measure the distance at which we
observe objects in the universe. The cosmic redshift $z$ is given by the
relation
\begin{equation}
    z = \frac{\lambda_{obs}}{\lambda_{em}} - 1,
\end{equation}
where $\lambda_{obs}$ is the observed wavelength we measure here on earth and
$\lambda_{em}$ is the wavelength emitted by the observed object. by combining this with the doppler formula
$\frac{\lambda_{obs}-\lambda_{em}}{\lambda_{em}} = \frac{v}{c}$, one gets
\begin{equation}
    z = \frac{v}{c},
\end{equation}
where $v$ is the velocity of the observed object relative to the observer. The
velocity of an object due to the expansion of the universe is given by Hubbles law
\begin{equation}
    v = H_0 d_p,
\end{equation}
where $d_p$ is what is known as proper distance and $H_0$ is the Hubble
parameter at present time. The Hubble parameter measures the expansion rate of
the universe as a function of time. (Kanskje prate om verdien av H0). Proper distance can be interpreted as the length measured if one
was to use a ruler between two objects $a$ and $b$ at a fixed time $t$ taking
into account the fact that our universe is expanding. Before introducing how proper distance is calculated it can be
useful to talk about the concept of comoving distance first. If we choose a time
$t=t_0$, which is present time, and introduce a coordinate system at this time,
we can choose an arbitrary point to center ourselves in in this coordinate
system. If we apply a ruler in this coordinate system and measure the distance
from our point to another arbitrary point, this is the comoving distance $r$.
The comoving distance remains constant through time as the universe expands. The
comoving distance is simply the distance measured in a coordinate system at
$t=t_0$ (dette er kanskje en krokete formulering $d_c$). The proper distance on the other hand factors in the expansion of the
universe through the scale factor $a(t)$. The proper distance is given as
\begin{equation}
    d_p = \int_{r_0}^{r_1}a(t)rdr,
\end{equation}
Where $a(t)$ is a quantity known as the scale factor. The scale factor parametrises the relative expansion of the
universe. It is defined in such a way that $a_0$, subscript $0$ meaning
$a(t=t_0)$ where $t_0$ is present time, is equal to one. This definition of
proper distance also implies that comoving distance is the proper distance at $t=t_0$.
With these properties in place we can now see how redshift, which is a
measurable quantity, can be used to measure the real distance to objects in our universe.\\

Another measurable measurable quantity is flux emitted from an astronomical object.
The flux $F$ is defined as
\begin{equation}
    F = \frac{L}{4\pi R^2},
\end{equation}
where $L$ is the luminosity and $R$ is the distance to the object. Luminosity is a measure
of the absolute electromagnetic power emitted from an astronomical object. The flux however, measured in $W/m^2$, is simply the luminosity
divided by the area of a shell with a radius at which we observe the object, assuming the radiation is isotropically emitted. Some astronomical phenomena, like a type 1a supernova, have a fixed luminosity
due to the characteristics of the process that creates it. This is called a standard candle. By measuring the observed flux from such a phenomena, one can then
calculate the the luminosity distance as 
\begin{equation}
    d_L = \sqrt{\frac{L}{4\pi F}}.
\end{equation}
(Kanskje si noe mer om svakheter med denne.)


\subsection{Modelling the evolution of the universe and the $\Lambda$CDM model}
The cosmological principle states that on sufficiently
large scales the universe is isotropic and homogenous. This means the universe
looks the same in every direction for every observer at any point in space for
sufficiently large scales. By sufficiently large scales one means
distances roughly $100$Mpc or larger \cite[p.~12]{ryden2017introduction}. This has been proved to be a good
assumption by surveys like the Sloan sky survey (svak formulering. Trenger ogsaa
kilde). Figure \ref{fig:sdssmap} shows a sky map from a survey released by the
SDSS project.\\
\begin{figure}[htbp]\label{fig:sdssmap}
    \includegraphics[scale=0.4]{orangepie.jpg}
    \caption{The SDSS (Sloan Digital Sky survey) map of the universe. Each dot represents a galaxy and the color scale represents the local density. Downloaded from the SDSS website. Image Credit: M. Blanton and SDSS (Vet ikke hvordan jeg skal sitere denne)}
\end{figure}
The fact that the universe was homogeneous and isotropic was an essential assumption
for a particular solution of the Einstein field equation. 
This particual equation, first published by Albert Einstein in $1915$ \cite{Einstein1915} in relation to his theory of general relativity, is given as
\begin{equation}\label{eq:einstein}
    G_{\mu\nu}=8\pi GT_{\mu\nu}.
\end{equation}
This equation relates $G_{\mu\nu}$, which is the Einstein tensor in which describes the curvature of spacetime, with the energy momentum tensor
$T_{\mu\nu}$, which describes the energy content of the universe.
The Friedman-Lemaître-Robertson-Walker metric, which assumes the universe is
homogenous, isotropic and expanding provides an analytical solution to equation \ref{eq:einstein}.
The metric is a tool that allows us to calculate the physical distance in a coordinate system (Dette er kanskje veldig vagt).
For an example for the Cartesian coordinate system we have the metric 
\begin{equation}
    g_{ij}=
    \begin{bmatrix}
        1 & 0 \\
        0 & 1 
    \end{bmatrix}.
\end{equation}
The definition of the line element $ds^2$ in this case, is given as
\begin{equation}
    ds^2 = \sum_{i,j=0}^1g_{ij}dq^idq^j,
\end{equation}
where $q=(x, y)$ is a vector representing the spatial dimensions and $i$ and $j$ represents the indices.
This gives the familiar line element $ds^2=dx^2+dy^2$.
For a flat universe the Friedman-Lemaître-Robertson-Walker metric for four dimensional space time takes the following form
\begin{equation}
    g_{\mu\nu}=
    \begin{bmatrix}
        1 & 0 & 0 & 0\\
        0 & -a^2 & 0 & 0\\
        0 & 0 & -a^2 & 0\\
        0 & 0 & 0 & -a^2 
    \end{bmatrix}.
\end{equation}
The line element is now given as 
\begin{equation}
    ds^2 = \sum_{\mu,\nu=0}^3g_{\mu\nu}dx^\mu dx^\nu,
\end{equation}
where $dx^0=dt$ and the remaining three indices are for the three spatial coordinates.
This gives us the following line element
\begin{equation}
    ds^2 = dt^2 -a^2(dx^2 + dy^2 + dz^2).
\end{equation}
The energy momentum tensor $T_{\mu\nu}$ for a perfect isotropic fluid takes the
form
\begin{equation}
    T_{\mu\nu}=
    \begin{bmatrix}
        -\rho & 0 & 0 & 0\\
        0 & -a^2P & 0 & 0\\
        0 & 0 & -a^2P & 0\\
        0 & 0 & 0 & -a^2P 
    \end{bmatrix},
\end{equation}
where $P$ is the pressure.
Using this metric together with the Einstein field equations results in the important
equations known as the first and second Friedman equations respectively as
\begin{equation}\label{eq:F1}
    \frac{\dot{a}^2}{a^2} = \frac{8\pi G\rho}{3}
\end{equation}
and
\begin{equation}\label{eq:FII}
    \frac{\ddot{a}^2}{a^2} = -\frac{4\pi G}{3}(\rho + \frac{3p}{c^2}).
\end{equation}
I refer the reader to \cite[ch. 2]{Dodelson:1282338} for an explanation on how to derive the Friedmann equations starting with the Friedman-Lemaître-Robertson-Walker metric.
These differential equations relate evolution of the scale factor $a$ with the
following properties of the universe density $\rho$, pressure $p$ and gravitational
constant $G$. In these equations $\dot{a}$ and $\ddot{a}$ dot denotes
derivative and double derivative with respect to time, meaning that $a=a(t)$ is a
function of time. \\

In modern cosmology, the dominating model for describing the universe is the $\Lambda$CDM model. Its name is an abbreviation for what 
is considered the main energy contributions governing the expansion of the universe. The cosmological constant $\Lambda$ represents dark energy, CDM is an abbrevation for cold dark matter
and lastly we have ordinary matter which is what we interact with in our
everyday lives. The current estimates suggest that approximately $69\%$ of the
universe consists of dark energy while the remaining $31\%$ is attributed to
dark matter at around 
$27\%$ and $4\%$ for regular matter leaving only trace contributions from other
energy contributing factors such as photons ($\gamma$) and neutrinos ($\nu$)
\cite{planckparameters}. When modelling the universe these quantites
enter into 
what is called density parameters $\Omega_i$, where $i$ represents a certain type of energy contribution to the universe i.e dark matter or CDM. The density parameter is defined as
\begin{equation}\label{eq:densityparameter}
    \Omega_{i,0} = \frac{\rho_{i,0}}{\rho_{c,0}},
\end{equation}
where $\rho_{i,0}$ is the density of a given energy contribution and $\rho_{c,0}$ is
the critical density of the universe. Subscript zero indicating these quantities
measured at present time. The critical density is the density at
which gravity counteracts the expansion and expansion will eventually stop
(Dette er jeg veldig usikker på om er riktig å si). The critical density at a
given time is given as $\rho_c=\frac{3H(t)^2}{8\pi G}$. For a flat universe we have
\begin{equation}
    \sum_i \Omega_i = 1.
\end{equation}
By combining both Friedmann equations (equations \ref{eq:F1} and \ref{eq:FII}),
one can derive a following relation for the density $\rho$
\begin{equation}\label{eq:rhoevolution}
    \dot{\rho}=-3\frac{\dot{a}}{a}(\rho+\frac{p}{c^2}).
\end{equation}
By inserting the equation of state $p=w\rho c^2$, where $w$ is a constant,
into equation \ref{eq:rhoevolution} one will get the following differential
equation
\begin{equation}
    \dot{\rho}=-3\frac{\dot{a}}{a}(1+w)\rho.
\end{equation}
This equation can be integrated for an arbitrary $a$ and $\rho$ to present time
giving
\begin{equation}\label{eq:rho_i_evolution}
    \rho_i=\rho_{i,0}a^{-3(1+w_i)},
\end{equation}
where subscript $i$ is again added to denote a specific energy contribution to
the universe. The constant $w_i$ takes different values when considering the
different energy contributions. Giving. For CDM and baryons we have $w=0$, for
radiation $w=\frac{1}{3}$ and for the cosmological constant we have $w=-1$. By
taking into consideration equation \ref{eq:F1} and dividing both sides by $H_0^2$,
and recognize the critical density $\rho_{c,0}=\frac{3H_0^2}{8\pi G}$, one can
rewrite the first Friedmann equation as
\begin{equation}
    \frac{H(t)^2}{H_0^2}=\frac{\rho}{\rho_{c,0}}.
\end{equation}
By separating the density into its different components as $\rho=\sum_i\rho_i$
and using \ref{eq:rho_i_evolution} for the different components, one can rewrite
the first Friedmann equation using equation \ref{eq:densityparameter} as
\begin{equation}
    H(t)=H_0\sqrt{(\Omega_{CDM,0} + \Omega_{b,0})a^{-3} + \Omega_{r,0}a^{-4} + \Omega_{\Lambda,0}},
\end{equation}
Where $b$ represents baryonic matter and $r$ represents radiation.

\subsection{Einstein-de Sitter model}
While the $\Lambda$CDM model is the leading model for describing the evolution
of our universe, other universe models, being simpler to work with, will also
prove beneficiary for approximating and describing multiple phenomena. One of
these models is the Einstein-de Sitter model \cite{1932PNAS...18..213E} proposed
by Albert Einstein and Willem de Sitter in 1932. This model contains only matter
making $\Omega_m=1$. After learning of the expansion
of the universe from the observations of Edwin Hubble, Einstein removed the
cosmological constant from his equation as this was first proposed to keep the
universe static. After the discovery that the expansion of the universe is
accelerating \cite{Goldhaber_2009}\cite{Filippenko_1998} the cosmological
constant was reintroduced. While this universe model is long discarded it is
still usefull as it is easy to work with in an analytical framework and provides
good approximations for different applications. In a universe containing only
matter, the first Friedmann equation with $w=0$ takes the form
\begin{equation}
    \frac{\dot{a}^2}{a^2} = \frac{8\pi G}{3}\rho_{m,0}a^{-3}.
\end{equation}
By multiplying with $H_0^2/H_0^2$ and remembering that $\rho_{c,0}=\frac{3H_0^2}{8\pi
G}$, one gets
\begin{equation}
    a\dot{a}^2=H_0^2\frac{\rho_{m,0}}{\rho_{c,0}}.
\end{equation}
Since we consider a flat universe conatining only matter we have
$\frac{\rho_{m,0}}{\rho_{c,0}}=\Omega_{m,0}=1$. Taking the square root of both
sides, we get
\begin{equation}
    a^{1/2}\dot{a}=H_0.
\end{equation}
This gives us a differential equation in which we can integrate as
\begin{equation}
    \int_{a_0}^a a^{1/2}\mathrm{d}a=H_0\int_{t_0}^t \mathrm{d}t,
\end{equation}
which gives the solution
\begin{equation}
    \frac{2}{3}(a^{3/2}-a_0^{3/2})=H_0(t-t_0).
\end{equation}
By choosing time $t=0$ at with $a=0$ and using that $a_0=1$, we get
\begin{equation}
    \frac{2}{3}=H_0t_0.
\end{equation}
By imposing this boundary condition we find that
\begin{equation}
    a(t)=\frac{3H_0}{2}(\frac{t}{t_0})^\frac{2}{3}.
\end{equation}

\subsection{Growth of matter perturbations in linear perturbation theory.}\label{sec:linpert}
As i have previously stated, the universe is isotropic and homogenous. This
however was for sufficiently large scales at around $100$ Mpc and larger. On
smaller scales the universe is not isotropic and homogenous. A single galaxy,
for example, is denser than the intergalactic medium that separates it from other
galaxies. The anisotropy in temperature of the cosmic microwave background (CMB), as measured by
the Planck satellite and its predecessor CMB experiments, is measured to be
around $\Delta T/T=10^{-5}$. This shows that at around redshift $z=1000$, we had
relatively small fluctuations while we today observe galaxy clusters which has around
$200$ times larger density than the average density of an equal sphere in the
universe \cite[p.~342]{schneider2006extragalactic}. This suggests that the original density
perturbations grow over time. The small structure deviations at early times evolve through what is known as gravitational insatbility.
Regions with higher density will attract more material through gravity giving a flow of material towards dense regions making them even more dense.
This makes an already irregular distribution of matter become even more irregular with time, and will evolve from the small fluctuations at time of the CMB
to the dense sctructures we observe in our universe today.
To describe this evolution, one defines the relative density
contrast as
\begin{equation}\label{eq:overdensity}
    \delta(\vec{r}, t) \equiv \frac{\rho(\vec{r}, t) - \bar{\rho}(t)}{\bar{\rho}(t)}.
\end{equation}
Here $\bar{\rho}(t)$ denotes the mean matter density of the whole universe at
time $t$, while $\rho(\vec{r}, t)$ denotes the local density at position
$\vec{r}$ at time $t$.\\

The growth of these perturbations can, on scales substantially smaller than the
hubble radius, which is the radius of the observable universe, be described in
the framework of linear perturbation theory. On these scales Newtonian gravity
is sufficient to describe the nature of structure growth. With the approximation
that the universe only consists of pressureless matter, which is the case for
cold dark matter in which is the main constituent of the matter in the universe, we
will model the matter dsitribution as a pressureless fluid. The equations of
motion for such a fluid is governed by the equations
\begin{equation}\label{eq:continuuity}
    \frac{\partial \rho}{\partial t} + \nabla\cdot(\rho \vec{v})=0,
\end{equation}
\begin{equation}\label{eq:eulereq}
    \frac{\partial \vec{v}}{\partial t} + (\vec{v}\cdot\nabla)\vec{v}=-\frac{\nabla P}{\rho}-\nabla \Phi,
\end{equation}
\begin{equation}\label{eq:poisson}
    \nabla ^2\Phi=4\pi G\rho -\Lambda.
\end{equation}
Here $\vec{v}=\vec{v}(\vec{r},t)$ is the velocity field of the fluid and $\Phi=\Phi(\vec{r},t)$
is the newtonian gravitational potential. $P$ is the pressure of the fluid. Equation \ref{eq:continuuity} is the
continuuity equation which tells us that the density changes with the flow of
mass. Equation \ref{eq:eulereq}, which is the Euler equation, describes the
behaviour of the fluid under the influence of an external force. The last
equation, equation \ref{eq:poisson}, is the Poisson equation which relates the
gravitational potential to the density field. These equations are solvable in
the limit $\vert\delta\vert \ll 1$, and can then be used to derive an expression
for relatively small density contrasts, which can be used to describe the
evolution of density perturbations on large scales (SJekk opp om dette er
riktig.). We can model small perturbations to first order by substituing $\rho =
\rho_0 + \delta \rho$, $\vec{v} =\vec{v_0} + \delta \vec{v}$, $\vec{v} =\vec{v_0}
+ \delta \vec{v}$, $P = P_o + \delta P$ and $\Phi = \Phi_0 +\delta\Phi$. Here
$\delta$ is used to assign small perturbations to the physical quantities and should not be
confused with the previously defined density contrast. The velocity also has to
take into account the expansion of the universe in addition to the peculiar
velocity of the particles giving $\vec{v} = H(t)\vec{r} + \vec{v}_{pec}$. These
equations can be reduced to 
\begin{equation}
    \frac{\partial^2 \delta}{\partial t^2} + 2H(t) \frac{d \delta}{dt}=4\pi G\bar{\rho}\delta.
\end{equation}
\cite[p.~345]{schneider2006extragalactic}. This equation only contains time derivatives, and all the coefficients does not
depend on $\vec{x}$. Therefore $\delta(\vec{x}, t)$ can then be expressed a spatial and time
dependent expression on the form 
\begin{equation}
    \delta(\vec{x}, t) = D(t)g(\vec{x}).
\end{equation}
Here $g(\vec{x})$ is an arbitrary function of the spatial comoving coordinate
$\vec{x}$, and $D(t)$ satisfies
\begin{equation}
    \frac{\partial^2 D}{\partial t^2} + 2H(t) \frac{d D}{dt}=4\pi G\bar{\rho}D.
\end{equation}
This equation has two solutions. One solution is strictly increasing and one is
strictly decreasing. As time evolves the decreasing solution will become negligible
and the increasing solution will dominate. This solution, denoted as $D_+(t)$,
is called the growth factor and determines the amplitude of structure growth as
a function of time. Since $t\propto a$, one can then model the growth factor for
a given cosmological model. We will lastly define the function
\begin{equation}\label{eq:growthfac}
    f(a) \equiv \frac{d log D_+}{d log a},
\end{equation}
to quantify the relationship between the growth of structure and the expansion
of the universe.
\subsection{Spherical collapse models}
Linear perturbation theory has its limitations. In particular it will not be
able to accurately describe the formation of dark matter halos or galaxy
clusters. The evolution of density contrasts of order higher than unity requires
different techniques to model in a reasonable way. The spherical top-hat model
is one of the most fundamental frameworks to analytically model the non linear
collapse of overdensites in the universe.
\begin{figure}\label{fig:tophat}
    \tdplotsetmaincoords{80}{20}
    \begin{tikzpicture}[tdplot_main_coords]
        \coordinate (O) at (0,0,0);
        \fill[blue!50,opacity=0.5] (1,1,0) -- (10,1,0) -- (10,10,0) -- (1,10,0) -- cycle;
        \node [cylinder, shape border rotate=90, draw,minimum height=3cm,minimum
        width=4cm, fill=blue!50, opacity=0.30] at (5,6,1) (A) {};
        \draw[dashed]
        let \p1 = ($ (A.after bottom) - (A.before bottom) $),
            \n1 = {0.5*veclen(\x1,\y1)-\pgflinewidth},
            \p2 = ($ (A.bottom) - (A.after bottom)!.5!(A.before bottom) $),
            \n2 = {veclen(\x2,\y2)-\pgflinewidth}
         in
         (A.before bottom) arc [start angle=0, end angle=180,
         x radius=\n1, y radius=\n2];

        \draw[->] (O) --++ (1,0,0) node[below] {$x$};
        \draw[->] (O) --++ (0,1,0) node[below] {$y$};
        \draw[->] (O) --++ (0,0,1) node[right] {$\rho$};    
    \end{tikzpicture}
    \caption{A simple illustriation of the spherical top hat model.}
\end{figure}
The spherical top-hat model gets its name from how its density profile is
defined. For the spherical top-hat model we have
\begin{equation}
    \rho(\vec{r},t) =
    \begin{cases} 
        \bar{\rho}(1+\delta_i), & \vert\vec{r}-\vec{r}_c\vert\leq R \\
        \bar{\rho}, & \vert\vec{r}-\vec{r}_c\vert \textgreater R,
     \end{cases} 
\end{equation}
where $\vec{r}_c$ is the center of the overdensity and $R$ is the radius.
$\delta_i$ is a factor determining the size of a given overdensity. Figure
\ref{fig:tophat} gives a simple graphical illustration of the model, hence why
it has the name spherical top-hat model. For a spherical shell with the given
density we have the mass
\begin{equation}
    M=\frac{4}{3}\pi R^3\bar{\rho}(1+\delta_i).
\end{equation}
The evolution of the radius $R$ can be modelled through Newtons second law as
\begin{equation}
    \frac{d^2R}{dt^2}=-\frac{GM}{R^2}.
\end{equation}
By multiplying both sides of this equation with $\frac{dR}{dt}$ and integrating
one gets
\begin{equation}\label{eq:kinpot}
    \frac{1}{2}\frac{dR}{dt}-\frac{GM}{R}=E.
\end{equation}
Here it is worth noting that the left hand side of \ref{eq:kinpot} is the sum of
the kinetic and potential energy. Therefore the integration constant $E$ can be
interpreted as the total energy. It can be shown that, for
$E<0$, this equation has the parametrised solution
\begin{align}
    R&=A(1-\mathrm{cos}\theta)\\
    t&=B(\theta-\mathrm{sin}\theta)\\
    A^3&=GMB^2.
\end{align}
\cite[p.~79]{peebles1980}. By setting an initial radius $R_i$ at time $t_i$ this
solution tells us that a shell expanding with the background evolution of the
universe will reach eventually slow down and reach a maximum radius at
$\theta=\pi$. This is called turn-around. This point is called turn around and now the sphere will begin to
collapse. This solution also suggests that at $\theta=2\pi$ the sphere would
collapse to a point. This is not the case as something known as virialization
occurs before this happens.\\\indent
The virial theorem describes the relation between the potential and kinetic
energy in dynamical equilibrium in a set of collisionless particles effected by
a potential. Individual particles in the gas will reach $R=0$ and eventually
shoot outwards again. The whole system will stabilize when it has virialized. The virial theorem states that
\begin{equation}\label{eq:virial}
    \vert U\vert=2K,
\end{equation}
where $U$ is the potential energy and $K$ is the kinetic energy.
At the radius of turn around, the kinetic energy is zero. As the cloud collapses
teh kinetic energy will increase and eventually it will virialize when the
conditions of equation \ref{eq:virial} is fulfilled.
\subsection{Two point correlation functions and the matter power spectrum.} \label{sec:corrtheory}
Matter in the universe is not randomly distributed. As a result of how structure in the universe evolves, galaxies for example,
does spread out randomly but they gather in groups or clusters. This means that given a random point in space, the probability of finding
a galaxy in the vicinity of that point is higher if the randomly chosen point by chance landed on a galaxy. The probability of finding a galaxy
at point $\vec{x}$ is not independent of wether there is a galaxy at an neighbouring point $\vec{y}$. This is a statistical property of the distribution
of galaxies and can be described by a two point correlation function. The two point correlation function for galaxies $\xi_{g}(r)$ describes the excess probability
over random for finding two galaxies separated by a distance $r$. The correlation is related to the density contrast as
\begin{equation}
    \xi(\vert\vec{r_1}-\vec{r_2}\vert)=\langle\delta(\vec{r_1})\delta(\vec{r_2})\rangle.
\end{equation}

The matter power spectrum $P(k)$ provides a statistical description of the
distribution of matter in the universe, but in fourier space. Figure
\ref{matterpowerspec} shows the matter power spectrum inferred from a multitude
of cosmological surveys. The matter power spectrum describes the amplitude of 
the density contrast on different length scales $L=\frac{2\pi}{k}$, where $k$ is
the fourier wave number. The matter power spectrum is related to the two point
correlation function as a fourier transform given by
\begin{equation}
    P(k)=2\pi\int_0^\infty x^2\frac{\mathrm{sin}(kx)}{kx}\xi(x)dx.
\end{equation}
\begin{figure}[htbp]\label{fig:matterpowerspec}
    \includegraphics[scale=0.7]{matterpowerspec.pdf}
    \caption{The matter power spectrum inferred from different cosmological surveys. Image credit: "ESA and the Planck Collaboration" \cite{2020}.}
\end{figure}
Both of these quantites are statistical properties and are of high importance
when comparing models to observations or simulations. Since we only have one
universe to sample from when doing observations, when we compare with the
predictions from models a pure image comparison may not prove to give an
accurate picture of the performance of our models(Må kanskje forklares bedre). It is therefore important to
note that two universes are considered identical if they inhabit the same
statistical properties. A simulation or calculation of structure growth may
not look identical to anything anywhere in the universe (if it is not infinately
large), but if the statistical properties are the same, then the models are
correct. This makes the power spectrum and two point correlation function
important vital tools when studying the universe.

\subsection{The cosmic web}
On large scales the universe is isotropic and homogenous. However this does not
mean that every galaxy or cosmological particle is distributed randomly. As
previously discussed matter tends to flow towards overdensities. Over time many
of these overdensities will collapse into large dark matter halos. Dark matter
halos contain subsctructures and sub halos where baryonic matter can form
galaxies. The massive halos are bound together by matter in what resembles a
web structure known as the cosmic web. The cosmic web is made up of large
filaments of baryonic and dark matter and in between there are large voids with
very low density. Figure \ref{fig:millenium} shows the Millenium simulation with
$10^{10}$ dark matter particles at different redshifts $z=0$ and $z=18.3$ corresponding to
$t=0.21$Gyr and $t=13.6$Gyr after the big bang respectively.
\begin{figure}[htbp]\label{fig:millenium}
    \subfigure[]{\includegraphics[width=0.5\textwidth]{figures/milleniumz0.jpg}}\hspace{1em}%
    \subfigure[]{\includegraphics[width=0.5\textwidth]{figures/milleniumzoomz0.jpg}}
    \subfigure[]{\includegraphics[width=0.5\textwidth]{figures/milleniumz18.jpg}}\hspace{1em}%
    \subfigure[]{\includegraphics[width=0.5\textwidth]{figures/milleniumzoomz18.jpg}}
    \caption{Figure showing snapshots of the Millennium simulation with $10^{10}$ particles
    for two different redshifts. Figures a) and a) is taken at $z=0$ while
    figures c) and d) is taken at $z=18.3$. Image credit: Downloaded from the
    website of the Millennium Simulation\cite{Millennium} (\url{https://wwwmpa.mpa-garching.mpg.de/galform/virgo/millennium/}).}
\end{figure}
Here one can cleary see that as time evolves the matter coalesces from a
relatively even distribution of matter into a web like structure with filaments
and voids. The same pattern is also visible in the SDSS sky survey shown in
figure \ref{fig:sdssmap}.\\\indent
An important phenomena for the matter to coalesce into this weblike structure is
baryonic acoustic oscillations (BAO). Baryonic acoustic oscillations originate
from the dense hot plasma that the universe consisted of after the big bang.
During this time photons and baryonic matter was coupled together. While dark
matter is collisionless, the baryon-photon plasma on the other hand is not and
is subject to pressure forces. Gravity will cause matter to flow towards
overdensities. The collisionless dark matter will clump together and form
potential wells, while the baryonic matter is subject to outward pressure as it clumps
together. While the photons and baryons are coupled together they move with the
sound speed which in this plasma is roughly half the speed of light. The
outwards pressure will caus riples of baryonic matter flowing outwards from the
dark matter overdensities. At approximately $z\approx 1000$ baryons and photons
decoupled. In cosmological history this period is called the epoch of
recombination. Due to the expansion of the universe the interaction rate between
photons and baryonic matter experienced decreased drastically. This caused the
rate at which photons scattered of baryonic matter to halt and neutral hydrogen was formed.
As baryonic matter was no longer bound to photons this also caused the sound
speed to drop eventually causing the riples to stop expanding. This lead to
shells of baryonic matter eventually attracting
portions of dark matter away from the overdensities in the center of the
potential wells formed before recombination. These riples are not directly
observed but can be inferred from measuring the correlation function for
galaxies. The correlation function contains a bump at the characteristic length
scale $r_s$. This shows that the baryonic acoustic influential on the distribution of
matter in the universe today.



\subsection{Real- vs Redshiftspace}
As mentioned earlier in section \ref{sec:sec_distance}, due to the fact that the
universe is expanding, galacies we observe in the universe on cosmological scales
move away from us and appear redshifted. In addition galaxies also have peculiar
velocities, which is their velocity when subtracting the hubble flow. This results in an effect that when observing galaxies
as a function of redshift, their spatial distribution appear squashed. As most
cosmological surveys of galaxies are done by measuring their redshift,
understanding how this effects their spatial distribution is crucial.
We will now denote realspace coordinates by subscript $r$ and redshiftspace
coordinates by subscript $s$. By taking into account the peculiar velocity of
galaxies and the hubble flow, we can express their redshift as a sum of two
terms
\begin{equation}\label{eq:losrs}
    cz=H_0d_p+v_{pec}.
\end{equation}
Here the first term represents the hubble flow and the second term is the
peculiar velocity along the line of sight. For most galaxies, only the redshift
can be measured and not the proper distance. Therefore, by dividing equation
\ref{eq:losrs} by $H_0$, it is usefull to define
the redshift distance $s$ as
\begin{equation}
    s = \frac{cz}{H_0}=d_p+\frac{v_{pec}}{H_0}.
\end{equation}
\\

The peculiar velocity of galaxies can also have a profound effect on how their
spatial distributions appear in redshiftspace. Matter will cluster and move
towards overdensities. This means that the velocity component of galaxies in a
cluster will essentialy point towards the center of the galaxy cluster. This
will give an effect that when viewing a galaxy cluster, the galaxies closest to
us in realspace will have their peculiar velocity move away from the observer.
From equation \ref{eq:losrs} one can see that this will give an additional
contribution in addition to the hubble flow and the measured redshift will be
higher. Galaxies on the opposite side of the cluster as seen by the observer
will have their peculiar velocity point towards the observer. This will cause
the peculiar velocity to contribute with a blueshift, and likewise the measured
redshift will appear smaller than if one could only measure the redshift due to
the hubble flow. This effect will cause galaxy cluster to appear squashed in
redshiftspace. This effect will also cause the overdensity to appear much larger
in the center of a collapsed object than in realspace. This is illustrated in
figure \ref{fig:rsddistortion}\\

\begin{figure}[htbp]\label{fig:rsddistortion}
    \begin{tikzpicture}
        \draw (5,0) circle (3cm) node[above] {Realspace};
        \draw (13,0) ellipse (3cm and 1cm) node[above] {Redshiftspace};
        \draw[line width=0.5mm][->] (5,3cm) -- (5,1.5cm);
        \draw[line width=0.5mm][->] (5,-3cm) -- (5,-1.5cm);
        \draw[line width=0.5mm][->] (5+1.8cm,0) -- (5+3.5cm,0);
        \draw[line width=0.5mm][->] (5+7.8cm,0) -- (5+5.8cm,0);
    \end{tikzpicture}
    \caption{Figure illustrating a collapsing halo in realspace and redshiftspace. The vector pointers represent the direction of the peculiar velocity. In this illustration both circles are meant to contain the same number density of galaxies thereby giving the apparent overdensity appear much higher in redshiftspace than in realspace. The observer sitting at the bottom of the page with the line of sight direction being parallell to the plane spanned by the circle and the ellipse.}
\end{figure}
The effect of redshiftspace distortions causes the correalation function and thereby
matter powerspectrum to be slightly altered when observed in redshiftspace. In
fourier space, the overdensity recieves a correction term as
\begin{equation}
    \delta_s(\vec{k})=(1+\beta\mu)\delta(\vec{k})
\end{equation}
\cite[p.~279]{Dodelson:1282338}. Here $\mu=\vec{z}\cdot\vec{k}$ is the angle
between the line of sight and the fourier wave vector. The factor $\beta$ can be
seen as the growth factor $f$ introduced in equation \ref{eq:growthfac} from
section \ref{sec:linpert}. However since the overdensity of galaxies does not
necesarilly follow the overdensity of dark matter, in which linear perturbation
theory is derived, it is scaled by a bias factor
\begin{equation}\label{eq:bias}
    b=\frac{\delta_g}{\delta_{dm}},
\end{equation}
where subscript $g$ is for galaxies and $dm$ is for dark matter. We then have
that
\begin{equation}\label{eq:beta}
    \beta=f/b.
\end{equation} 
\section{Cosmological N-body simulations}
Analytical approaches, such as linear perturbation theory has its limits.
Especially when taking into consideration small scale interactions between
galaxies. Using linear perturbation theory, gravitiational interactions on small scales can not be described in
enough detail to accurately compare observational data to theoretical
predictions. Therefore numerical simulations of structure formation is an
important tool when studying the properties of the universe. Numerical
simulations have been crucial for establishing $\Lambda CDM$ as the standard
model for cosmology as through simulations it has become possible to separate
the predictions of different models when comparing models to observations
\cite[p.~361]{schneider2006extragalactic}. The first cosmological N-body
simulation was conducted by \cite{PeeblesPJE1970SotC}, where the equations of
motion for $300$ particles were solbed to study the formation of galaxy clusters. Due to the significant increase in
computer power over the last few decades, one has been able to perform
increasingly more detailed numerical simulations where one can afford to
increase both the spatial and temporal dimension in the simulations.\\

When simulating large scale structure formation in the universe, it is mostly
sufficient to take into account dark matter, as this is the primary contributor
to the mass in the univers. The size of dark matter particles, which is still an
unobserved, and will be a lot smaller than the simulation volume. Therefore
all the dark matter particles in the simulation is represented by point
particles with mass $M$, where each point particle is a body of mass
representing multiple dark matter particles. One also has to restrict the
simulation volume of the simulation. The size of the universe is too large to
simulate in detail. It may also be infinite. Therefore one has to select a
simulation volume that is representative for the effects one wants to model. For
simulations where one wants to examine large scale structures, the simulation
volume is usually a box with size $L>200$Mpc as large scale structures are hardly present on
scales smaller than this \cite[p.~362]{schneider2006extragalactic}. Due to
restricting the simulation to a small box slice of the universe, problems with
the boundaries of the simulation volume will rise. If not treated, particles at
the boundary of the simulation volume will not be affected by gravity as many
neighbouring particles as particles in the middle of the simulations volum. To
account for the fact the universe should be isotropic and homogenous on large
scales one has to apply periodic boundary conditions. This means that the cube
is extended periodically. A particle leaving the volume in one end will reappear
in the other end. Particles will also interact gravitationally in the same
manner meaning that particles opposing sides of the simulation volume will
effect each other as if the simulation volume was extended by adding an
identical volume side by side.

\subsection{pair wise summation}
The simplest and most intuitive way of performing N-body simulations is by
modelling gravity as a force and applying Newtons second law to calculate the
acceleration vector of the particles. For every particle one calculates the
force acting on it from all other particles by summing over the contribution
from all other particles. The computational cost of this approach scales as
$N^2$, where $N$ is the number of particles in the simulation. This makes this
approach not feasible when including $N\gtrsim10^6$ particles (kanskje siter). However
interesting problems can be studied by not exceeding this number of particles. This problem can be formulated mathematically as
\begin{equation}\label{eq:newtongravacc}
    \frac{d^2r_i}{dt^2}=-G\sum_{i=1}^{N}\sum_{j\neq i}^N\frac{m_j}{\vert\vec{r_j}-\vec{r_i}\vert^3}(\vec{r_j}-\vec{r_i}),
\end{equation}
where we calculated the acceleration on the $i$th particle with position
$\vec{r}$ by summing over all particles $j\neq i$ with position $\vec{r_j}$.
When studying equation \ref{eq:newtongravacc}, one can see that if two particles
come very close numerical instability will occur in the form of a division by
zero giving an unreasonably high force calculation. To counteract this problem
force softening is introduced. A way to do this is to modify the denomenator of
equation $\ref{eq:newtongravacc}$ by including a small factor $\epsilon$
replacing $\Delta r_{i,j}=\vert\vec{r_j}-\vec{r_i}\vert$ with $\sqrt{\Delta
r_{i,j}^2+\epsilon^2}$(Siter klypin). We will now use the notation where the acceleration is
given as $a(t)=d^2r(t)/dt^2$ and the velocity $v(t)=dr(t)/dt$.
From equation \ref{eq:newtongravacc} we now have an expression for the acceleration of each object.
This gives us the the coupled system ODEs necessary to solve for the position as
\begin{equation}
    v(t)=\frac{dr(t)}{dt} \quad\mathrm{and}\quad a(t)=\frac{dv(t)}{dt}.
\end{equation}
We now discretize the time variable where $t_n$ is the time at a specific
time point where $t_n=t_0+n\Delta t t$ where $\Delta t$ is an incremental
timestep. By a Taylor expansion around $t_n$, one can approximate the time
solution at timestep $t=t_n+\Delta t$ giving
\begin{equation}
    r(t_n+\Delta t) = r(t_n) + \frac{dr(t_n)}{dt}\Delta t + \frac{1}{2}\frac{d^2r(t_n)}{dt^2}\Delta t^2 +\ldots
\end{equation}
and
\begin{equation}
    v(t_n+\Delta t) = v(t_n) + \frac{dv(t_n)}{dt}\Delta t + \frac{1}{2}\frac{d^2v(t_n)}{dt^2}\Delta t^2 +\ldots.
\end{equation}
By discreting the spatial variables in the same way as the time variable one can
write $r(t_n)=r_n$, $v(t_n)=v_n$ and $a(t_n)=a_n$ respectively for every particle.
Similariliy we have $r_{n+1}=r(t_n + \Delta t)$ etc.
By truncating the taylor expansion at the second order term one can then write
an approximate solution to our problem as
\begin{center}
\begin{itemize}
    \item $a_n=-\sum_{i=1}^{N}\sum_{j\neq i}^N\frac{Gm_j}{(\Delta
    r_{i,j}^2+\epsilon^2)^{2/3}}(\vec{r_j}-\vec{r_i})$
    \item $v_{n+1} = v_n + a_n\Delta t$
    \item $r_{n+1} = r_n + v_n\Delta t.$
\end{itemize}
\end{center}
This is done for every particle $i$ for every timestep $t+\Delta t$. This is one
of the simplest ways of this system of coupled ODEs and is called the forward
Euler method. This approach however is never used in practice as the error $\epsilon$ of
the solution, as can be seen from the taylor expansion, is of order
$\epsilon\sim\mathcal{O}(\Delta t^2)$. Another significant disadvantage with this method is
that when solving for physical systems, it will not conserve energy. A numerical
integration scheme well suited for gravitational problems where energy is
conserved is the Velocity verlet algorithm. This integration scheme is what is
known as symplectic, meaning that it conserves the energy of the system
\cite[p.~31]{holmes2007introduction}. The Velocity Verlet scheme updates its
velocity and position as
\begin{center}
    \begin{itemize}
        \item $r_{n+1} = r_n + v_n\Delta t+\frac{a_n}{2}\Delta t$
        \item $v_{n+1} = v_n + \frac{\Delta t}{2}(a_{n+1}+a_n).$
    \end{itemize}
\end{center}
This method, and its variations, that preserve energy over long time periods
makes it a preferred method of choice for astrophysical gravitational problems.
(Kanskje vise en figur som sammenlikner verlet of euler for å vise ikke konstant
energi hos euler. Kanskje bytte ut verlet integrasjon med leapfrog istedenfor.)
\subsection{Particle mesh}
Another way of simulating a box with a large volume of particles is it to apply
what is called a Particle Mesh code \cite{HockneyRW1981Csup} \cite{Fazio2309855}. This method is more computationally
effective as the number of calculations scales as $\propto N$. However most of
the information is stored on a $3D$ grid covering the whole simulation volume.
This will require more memory as the resolution of the grid increases. Poissons equation for gravity reads
\begin{equation}\label{eq:poissongrav}
    \nabla^2\Phi=4\pi G\rho.
\end{equation}
(Kanskje bruke den formen av likningen alle andre bruker med rho-rho bakgrunn)This equation can be solved for the gravitational potential $\Phi$ and updates
the force field as $\vec{F}=-\nabla\Phi$. This method is reliant on assigning a
density field to the simulation volume of individual particles as a
representation for $\rho$ in equation \ref{eq:poissongrav}. One popular method 
for this is what is known as cloud in cell. (Det
er her jeg har lest om det, men det er ikke de som har funnet det opp. Er det
riktig aa sitere slik da?). One wants to calculate the density $S$ at a given
distance given by the coordinate point $(x, y, z)$ from the particle. The density at a given point is given as
$S(x)S(y)S(z)$, where the cloud in cell scheme for assigning a value to $S$
reads
\begin{equation}
    S(x)=\frac{1}{\Delta x}
    \begin{cases}
        1, &\quad\text{if }\vert x \vert < \Delta x/2\\
        0, &\quad\text{otherwise},
      \end{cases}
\end{equation}
where $\Delta x$ is the cell size. The density assigned to the grid is defined
by a product of three weighting functions $W(\vec{r}_p -\vec{r}_{ijk}) =
W(x_p-x_i)W(y_p-y_j)W(z_p-z_k)$. Where subscript $p$ denotes particle position
and subscript $i,j,k$ denotes a given grid index. A given $W$ is given as
\begin{equation}
    W(x)=\int_{x_i-\Delta x/2}^{x_i-\Delta x/2} S(x_p - x^\prime)dx^\prime,
\end{equation}
where $m_p$ is the mass of particle $p$. We then have that the given density assigned to a given grid point for a grid
consisting of $N$ particles as is given as
\begin{equation}
    \rho_{ijk}=\sum_{p=1}^Nm_pW(\vec{r}_p-\vec{r}_{ijk}).
\end{equation}
With the density assigned one can solve the Poisson equation for gravity given
by equation \ref{eq:poissongrav}. This is done by using a fast fourier transform
(FFT) on the whole grid giving
\begin{align}
    \nabla^2\Phi&=4\pi G\rho\\
    -k^2 \hat{\Phi}&=4\pi G \hat{\rho},
\end{align}
Where $\hat{\Phi}$ and $\hat{\rho}$ denotes the fourier transform of the given
quantities. Dividing by $k^2$ and taking the inverse transform one solves the
equation for the gravitational potential as
\begin{equation}
    \Phi = \mathrm{IFFT}[-\frac{4\pi G \hat{\rho}}{k^2}].
\end{equation}
With the gravitational potential one can calculate the force and interpolate
forces to their respective particle positions in the grid. The particle mesh
method follows the same algorithm as the pairwise summation method, but differs
in the force calculation. Summarized the force calculation of the particle mesh
method consists of three steps:
\begin{itemize}
    \item Assign density to grid. \\
    \item Solve the Poisson equation given by equation \ref{eq:poissongrav}.\\
    \item Compute the force and apply to particles in the grid.
\end{itemize}
While the compuation time is significantly increased the
particle mesh method lacks in resolution. Increasing the resolution for a
particle mesh code means increasing the number of bins in each direction. For a
three dimensional grid this will quickly put constraints on the computer memory.

\subsection{Initial conditions}
When setting the initial conditions for cosmological simulations, one starts at
a high redshift. The initial distribution of the particles resembles a gaussian
random field with the theoretical matter power spectrum $P(k,z)$ for the given
cosmological model. Choice of timestep is also important. One has to find a
compromise between long computation time and accuracy between closely
interacting particles. 

\section{Statistics}
\subsection{Bayes theorem}
When studying and comparing our observations and simulations with theory one has
to quantify the probability of our models parameter space given the data we use.
I.e what parameters will make our model compare best with our observations or
simulations. Consider the probability $P(A, B)$, which is the probability for two events $A$ and $B$ occuring.
If the two events $A$ and $B$ are independent of each other, the probability is simply given as $P(A, B) = P(A)P(B)$.
This is however not the case if the events $A$ and $B$ are dependent on each other. Take for instance the probability
of drawing two knights in a row from a deck of cards. The first draw alters number of cards in the deck and will therefore affected
the probability of the other event. We write this as $P(A,B) = P(A)P(B\vert A)$. Here $P(B\vert A)$ denotes the probability of $B$ given the fact that 
event $A$ has occured. We can likewise write $P(A,B) = P(B)P(A\vert B)$. Now lets consider the event $A$ being the parameters of our model $\theta$ and
event $B$ being our data represented by $d$. We can then write $P(\theta,d) = P(\theta)P(d\vert \theta)=P(d)P(\theta\vert d)$. Rearranging terms we will get
what is known as Bayes theorem (sitere probability in physics)
\begin{equation}\label{eq:bayes}
    P(\theta\vert d) = \frac{P(d\vert \theta)P(\theta)}{P(d)}.
\end{equation}
$P(\theta\vert d)$ is what is known as the posterior and quantifies an
important question one would ask when studying cosmolgy or physics in general. What is the probability
of the parameters of the model being correct given the data one is studying? likewise $P(d\vert\theta)$ is the
probability of getting our data given the parameters we have testet. This is what is known as
the likelihood and is usually denoted as $\mathcal{L}(\theta)$. $P(\theta)$ is what is
known as the prior. This is used to quantify what we already know about our
parameter space. $P(d)$ is what is known as the evidence and acts as a
normalization factor. \\

In most practical examples examples our model contains multiple parameters we want to fit. Lets say we have a model
dependent on the parameters $\theta_0$ and $\theta_1$. We will then get the joint probability distribution $P(\theta_0, \theta_1)$
If we want to know the probability of $\theta_0$ independently of $\theta_1$ we can get what is know as the marginal distribution given by
\begin{equation}
    P(\theta_0)=\int_{-\infty}^{\infty}P(\theta_0, \theta_1)d\theta_1
\end{equation}

\subsection{Parameter estimation}
When comparing models with data, one has to find an expression to quantify the
agreement between the model and the data. An important statistical tool is what
is called the $\chi^2$ test. Given two functions $y^{model}_i$ and $y^{data}_i$,
representing the model and data respectively where $i$ is a given data point,
the $\chi^2$ becomes
\begin{equation}
    \chi^2=\sum_{ij}(y^{model}_i-y^{data}_i)C_{ij}^{-1}(y^{model}_j-y^{data}_j),
\end{equation}
where $C_{ij}$ is the covariance matrix if the data is correlated. If the data
is uncorrelated, the $\chi^2$ takes a simpler form
\begin{equation}
    \chi^2=\sum_{i}\frac{(y^{model}_i-y^{data}_i)^2}{\sigma_i^2},
\end{equation}
where $\sigma_i$ is the error estimate for the given data point $i$. This is a
useful tool for testing the scatter of our data. The likelihood function
$\mathcal{L}(\theta)=P(d\vert \theta)$ describes the probability of our dataset
$d$ being true given our parameters. By holding our dataset constant one can
maximize the likelihood function. In other words one can try to find the most
probable value for $\theta$. If one assumes the data is given by a gaussian
distribution, the likelihood function is defined by a multivariate gaussian
\begin{equation}
    \mathcal{L}(\theta) = \frac{1}{\sqrt{(2\pi)^n\vert C \vert}}\mathrm{exp}\big[{-\frac{1}{2}}\sum_{ij}(y^{model}_i-y^{data}_i)C_{ij}^{-1}(y^{model}_j-y^{data}_j)\big].
\end{equation}
From this we can see that the likelihood and the $\chi^2$ is related as
$\mathcal{L}\propto \mathrm{exp}[-\frac{1}{2}\chi^2]$. From this it is also evident that
maximizing the likelihood is the same as minimizing $\chi^2$. Using bayes theorem given by equation
\ref{eq:bayes} one can see that we now have an expression for the likelihood.
The evidence $p(d)$, which acts as a normalization factor is not important as we
want the relative probabilites of our parameters and therefore this term can be
ignored \cite{heavens2010statistical}. If one chooses a flat prior, one where
$p(\theta)$ is constant, one simply gets by adressing bayes theorem
\begin{equation}
    P(\theta\vert d) \propto \mathcal{L}(\theta).
\end{equation}
\subsection{Sampling and Metropolis hastings MCMC}
When dealing with distributions $P(\vec{\theta},\vec{d})$, where there is a
large amount of parameters $\vec{\theta}=(\theta_0, \theta_1,\dots, \theta_n)$,
a simple gridding algorithm can prove to be both time and memory consuming. If
we also take into account the fact that one may deal with gaussian
distributions, there are large areas of parameter space where the probability of
that particular configuration of parameters being true is slim to none.
Therefore by gridding one spends a significant amount of time exploring a relatively
uninteresting part of parameter space. Therefore it is useful to introduce the
concept of sampling and the Metropolis-hastings Markov-Chain Monte Carlo
method.\\

The Monte Carlo principle states that if you take
independent and identically distributed samples $\theta$ from an arbitrary
function $f(\theta)$, as the number of samples increases, the distribution of
samples will converge towards the function $f(\theta)$. This is expressed
mathematically as
\begin{align}
    P_N(\theta) &= \frac{1}{N}\sum_{i=1}^N\delta(\theta)\\
    &\rightarrow \lim_{n\to\infty}P_N(\theta)=p(\theta)
\end{align}
There is however a problem in that the function we want to calculate
$P(\theta\vert d)$ is a distribution that is unknown to us.(sjekk opp dette). We
know how to calculate it, but we cannot draw from it. Therefore one introduces a
proposal distribution $Q(\theta)$. This is a known function that one can easily
evualuate and pick samples from. The proposal distribution can be any function,
but it is usually a gaussian or any other function enclosing the parameter space
of the target function. The target function is the function one wants to
represent, but it may be compuationally ineffective to use a brute force grid
approach as many models contains a multitude of parameters. In our case the
density distrubtion is the likelihood or the posterior distribution. One of the
most popular algorithms for Monte Carlo Markov chain is the Metropolis hastings
algorithm. The basic idea behind the Metropolis hastings algorithm is to draw a
point $\theta^*$ from the proposal distribution $Q(\theta)$ based on the
previous draw and and compare an acceptance criterion with a probability which
can be drawn from a uniform distribution. The acceptance criterion in
Metropolis-Hastings is given by
\begin{equation}\label{eq:acceptance}
    \alpha = min(1, \frac{P(\theta^*)Q(\theta^*\vert \theta)}{P(\theta)Q(\theta\vert\theta^*)}),
\end{equation}
where $Q(\theta^*\vert\theta)$ represents the proposal distribution for a new
pick of parameters $\theta^*$ based on previous parameters $\theta$ and $P$ is
the target density distribution. After
drawing an initial value $\theta_0$ from the proposal distribution $Q$, the
Metropolis Hastings algorithm is a repeatable sequence of the following
\begin{itemize}
    \item Sample $\theta^*$ from $Q(\theta^*\vert\theta)$ \\
    \item Sample a number $u$ from the normal distribution \\
    \item if $u < \alpha$, where is given by $\alpha$ given by equation \ref{eq:acceptance}: \\
          \indent accept move and sample $\theta^*$\\
    \item else:\\
          \indent reject move and sample previous point $\theta$ instead.\\
    \item Repeat untill you have a sufficient amount of samples to represent the
    target distribution.
\end{itemize}
The idea with the acceptance criterion is that the algorithm will eventually
converge toward the peak of the target density distribution. If one just
accepted the next step without comparing to a random pick from the normal
distribution, the algorithm would just converge to the closest peak. By
comparing the acceptance criterion to a pick from the normal distribution, one
would also in some cases accept picks with a less probable state. this makes the
algorithm map the whole distribution, but it will quickly drift away from
spending too much time where the target density distribution is negligible. 
