\chapter*{Abstract}
\addcontentsline{toc}{chapter}{\numberline{}Abstract}
The large scale structure of the universe provides independent sets of experiments, in addition to established experiments such as CMB analysis and type Ia supernova, for extracting cosmological information. Due to redshift space distortions, anisotropy is introduced when observing the matter distribution in the universe. This allows for the measurement of cosmological parameters such as the energy density contributions and growth rate of structure.
Future cosmological surveys, such as Euclid, is expected to greatly increase the number of galaxies having their redshifts measured thereby increasing the size of datasets. In preparation for the analysis of future observational datasets, the methodology has to be thoroughly tested in order to extract as much cosmological information from the density field as possible.\\\indent
In this thesis, I derive and implement a model for the radial velocity of matter around cosmological filaments using linear perturbation theory. A proposed improvement to the radial velocity of matter surrounding voids is also implemented and tested. This model, together with the regular radial velocity derived from linear theory, is implemented into the void-galaxy cross-correlation function in redshift space using a linear bias approximation for the overdensity profile. In addition, the regular radial velocity proposed by linear theory is also implemented into the void-galaxy cross-correlation function in redshift space using an arbitrary dark matter overdensity profile. Using the MultiDark simulation suite, these models are compared with data from numerical calculations and their performance is assessed.  
\\\indent
I find the model to provide promising results when predicting the radial velocity of matter around cosmological filaments. Although the amplitude of the velocity predicted by linear theory is not entirely accurate, it manages to replicate the shape of the radial velocity derived from the dataset. More work is needed to see if this can prove to be a full-fledged addition to LSS studies. When applying fits to the void-galaxy correlation function in redshift space, I find the regular velocity model derived from linear theory to perform consistently better than the model with the proposed improvement when fitting the Alcock-Paczyński parameter $\epsilon$. Applying cuts to the dataset, including only voids with an effective radius $r>40$Mpc/h, I find the regular model to also perform better when fitting the growth rate $f/b$. Lastly, I find the parameter fits to the model using the dark matter overdensity profile to consistently provide accurate fits for $\epsilon$. However, the model does not provide accurate fits for the normalization of the dark matter overdensity profile even with cuts applied.
\\\indent