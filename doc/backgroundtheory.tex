\chapter{Background Theory}
\section{History of cosmology and how we model the universe}
\subsection{Distances in the universe and cosmological redshift}
Distances in an expanding universe can prove hard to fathom. In our everyday
lives we are used to measuring distances in meters and kilometers. The
average distance from the earth to the sun however is approximately $1.49\cdot10^{11}$m.
Allthough this distance on a cosmic scale is very short, the numbers in our
everyday units of measurement already start to become too big for us to have a
reasonable idea of how long this distance actually is. We have a clear idea of
what a meter and a kilometer is as we have to deal with them everyday. To make
things more manageable when measuring these large distances the average distance
from the earth to the sun has become its own distance unit known as the
Astronomical unit (AU). This unit is a great tool for measuring relative
distances in our own solar system, but our nearest star, Proxima Centauri, is
located at approximately $268394$AU away from our closest star the Sun.
Therefore it is convenient to introduce another unit of distance measurement known as light year. This unit with a value of $9.4607\cdot10^{15}$m is the distance at
which light travels in vacuum during one Julian year equal to $365.25$ days.
This gives us a more manageable account of the distance from the Sun to Proxima
Centauri as $4.244$ light years. With both the AU and the light year introduced
we can define the last unit of measuring distance, namely the parsec equal to $3.26$
light years. One parsec is equal to the distance at which $1$ AU subtends an angle
of one arcsecond (setningen er lik paa wikipedia og Barbara ryden). Most
cosmological distances are measured in parsec, abbreviated Pc, with either the
prefix mega or giga as cosmology deals mostly with intergalactic scales much
larger than the average distance between two galaxies.\\

Due to the fact that the universe is expanding, light traveling towards us from
far away objects gets redshifted. Since the expansion of the universe is
isotropic, i.e, the expansion rate of the universe is equal in all directions,
the redshift of this light can be used to measure the distance at which we
observe objects in the universe. The cosmic redshift $z$ is given by the
relation
\begin{equation}
    z = \frac{\lambda_{obs}}{\lambda_{em}} - 1,
\end{equation}
where $\lambda_{obs}$ is the observed wavelength we measure here on earth and
$\lambda_{em}$ is the wavelength emitted by the observed object. by combining this with the doppler formula
$\frac{\lambda_{obs}-\lambda_{em}}{\lambda_{em}} = \frac{v}{c}$, one gets
\begin{equation}
    z = \frac{v}{c},
\end{equation}
where $v$ is the velocity of the observed object relative to the observer. The
velocity of an object due to the expansion of the universe is given by Hubbles law
\begin{equation}
    v = H_0 d_p,
\end{equation}
where $d_p$ is what is known as proper distance and $H_0$ is the Hubble
parameter at present time. The Hubble parameter measures the expansion rate of
the universe as a function of time. (Kanskje prate om verdien av H0). Proper distance can be interpreted as the length measured if one
was to use a ruler between two objects $a$ and $b$ at a fixed time $t$ taking
into account the fact that our universe is expanding. Before introducing how proper distance is calculated it can be
useful to talk about the concept of comoving distance first. If we choose a time
$t=t_0$, which is present time, and introduce a coordinate system at this time,
we can choose an arbitrary point to center ourselves in in this coordinate
system. If we apply a ruler in this coordinate system and measure the distance
from our point to another arbitrary point, this is the comoving distance $r$.
The comoving distance remains constant through time as the universe expands. The
comoving distance is simply the distance measured in a coordinate system at
$t=t_0$ (dette er kanskje en krokete formulering $d_c$). The proper distance on the other hand factors in the expansion of the
universe through the scale factor $a(t)$. The proper distance is given as
\begin{equation}
    d_p = \int_{r_0}^{r_1}a(t)rdr,
\end{equation}
Where $a(t)$ is a quantity known as the scale factor. The scale factor parametrises the relative expansion of the
universe. It is defined in such a way that $a_0$, subscript $0$ meaning
$a(t=t_0)$ where $t_0$ is present time, is equal to one. This definition of
proper distance also implies that comoving distance is the proper distance at $t=t_0$.
With these properties in place we can now see how redshift, which is a
measurable quantity, can be used to measure the real distance to objects in our universe.


\subsection{Modelling the evolution of the universe and the $\Lambda$CDM model}
The cosmological principle states that on sufficiently
large scales the universe is isotropic and homogenous. This means the universe
looks the same in every direction for every observer at any point in space for
sufficiently large scales. By sufficiently large scales one means
distances roughly $100$Mpc or larger (Ryden). This has been proved to be a good
assumption by surveys like the Sloan sky survey (svak formulering. Trenger ogsaa kilde). This was an important assumption
for a particular solution of the Einstein field equation. This equation, first published by
Albert Einstein in $1915$ in relation to his theory of general relativity. This equation
relates the geometry of spacetime to the distribution of matter within it.
The Friedman-Lemaître-Robertson-Walker metric, which assumes the universe is
homogenous, isotropic and expanding provides an analytical solution to the
Einstein equation (kanskje forklare hva en metrikk er for noe). Using this
metric together with the Einstein field equations results in the important
equations known as the first and second Friedman equations respectively as
\begin{equation}
    \frac{\dot{a}^2 + kc^2}{a^2} = \frac{8\pi G\rho + \Lambda c^2}{3}
\end{equation}
and
\begin{equation}
    \frac{\ddot{a}^2}{a^2} = -\frac{4\pi G}{3}(\rho + \frac{3p}{c^2}) + \frac{\Lambda c^2}{3}.
\end{equation}
(Kanskje inkludere denne utledningen fra kosmo 2 i et appendiks)
These differential equations relate evolution of the scale factor $a$ with the
following properties of the universe density $\rho$, pressure $p$, cosmological constant $\Lambda$, gravitational
constant $G$ and the speed of light $c$ (ikke glem k). For a flat universe $k=0$
(Hva er et flatt univers). In these equations $\dot{a}$ and $\ddot{a}$ dot denotes
derivative and double derivative with respect to time, meaning that $a=a(t)$ is a
function of time. \\

In modern cosmology, the dominating model for describing the universe is the $\Lambda$CDM model. Its name is an abbreviation for what 
is considered the main energy contributions governing the expansion of the universe. The cosmological constant $\Lambda$ represents dark energy, CDM is an abbrevation for cold dark matter
and lastly we have ordinary matter which is what we interact with in our everyday lives. The current estimates suggest that approximately $69\%$ of the universe consists of dark energy while the remaining $31\%$ (Dette er hentet fra Planck 2018) is attributed to dark matter at around 
$27\%$ and $4\%$ for regular matter leaving only trace contributions from other energy contributing factors such as photons ($\gamma$) and neutrinos ($\nu$).(Kanskje vaere mer spesifikk her) When modelling the universe these quantites enter into 
what is called density parameters $\Omega_i$, where $i$ represents a certain type of energy contribution to the universe i.e dark matter or CDM. The density parameter is defined as
\begin{equation}
    \Omega_{i,0} = \frac{\rho_{i,0}}{\rho_{c,0}},
\end{equation}
where $\rho_{i,0}i$ is the density of the current energy contribution and $\rho_{c,0}$ is
the critical density of the universe. Subscript zero indicating these quantities
measured at present time.The critical density is the density at
which gravity counteracts the expansion and expansion will eventually stop
(Dette er jeg veldig usikker på om er riktig å si). The critical density at a
given time is given as $\rho_c=\frac{3H(t)^2}{8\pi G}$. For a flat universe we have
\begin{equation}
    \sum_i \Omega_i = 1.
\end{equation}
A given energy density evolves through time with the scale
factor as
\begin{equation}
    \rho_i=\rho_{i,0}a^{-3(1+w_i)},
\end{equation}
where $w_i$ is a constant depending on what energy density contribution one is
considering. Summing over all energy
density contributions and using the fact that the Hubble parameter can be
expressed as $H=\frac{\dot{a}}{a}$, one can rewrite the first friedman equation
as
\begin{equation}
    H(t)=H_0\sqrt{(\Omega_{CDM,0} + \Omega_{b})a^{-3} + \Omega_{r,0}a^{-4} + \Omega_{\Lambda,0}},
\end{equation}
Where $b$ represents baryonic matter and $r$ is radiation
(Her trengs det kanskje mer forklaring og er kanskje ikke den riktige versjonen
av likningen.) 

\subsection{Growth of matter perturbations in linear perturbation theory.}
As i have previously stated, the universe is isotropic and homogenous. This
however was for sufficiently large scales at around $100$ Mpc and larger. On
smaller scales the universe is not isotropic and homogenous. A single galaxy,
for example, is denser than the intergalactic medium that separates it from other
galaxies. The anisotropy in temperature of the cosmic microwave background (CMB), as measured by
the Planck satellite and its predecessor CMB experiments, is measured to be
around $\Delta T/T=10^{-5}$. This shows that at around redshift $z=1000$, we had
relatively small fluctuations while we today observe galaxy clusters which has around
$200$ times larger density than the average density of an equal sphere in the
universe (SChneider side 342.). This suggests that the original density
perturbations grow over time. To describe this evolution, one defines the relative density
contrast as
\begin{equation}
    \delta(\vec{r}, t) \equiv \frac{\rho(\vec{r}, t) - \bar{\rho}(t)}{\bar{\rho}(t)}.
\end{equation}
Here $\bar{\rho}{t}$ denotes the mean matter density of the whole universe at
time $t$, while $\rho(\vec{r}, t)$ denotes the local density at position
$\vec{r}$ at time $t$.\\

The growth of these perturbations can, on scales substantially smaller than the
hubble radius, which is the radius of the observable universe, be described in
the framework of linear perturbation theory. On these scales Newtonian gravity
is sufficient to describe the nature of structure growth. With the approximation
that the universe only consists of pressureless matter, which is the case for
cold dark matter in which is the main constituent of the matter in the universe, we
will model the matter dsitribution as a pressureless fluid. The equations of
motion for such a fluid is given by the equations
\begin{equation}\label{eq:continuuity}
    \frac{\partial \rho}{\partial t} + \nabla\cdot(\rho \vec{v})=0,
\end{equation}
\begin{equation}\label{eq:eulereq}
    \frac{\partial \vec{v}}{\partial t} + (\vec{v}\cdot\nabla)\vec{v}=-\frac{\nabla P}{\rho}-\nabla \Phi,
\end{equation}
\begin{equation}\label{eq:poisson}
    \nabla ^2\Phi=4\pi G\rho.
\end{equation}
Here $\vec{v}=\vec{v}(\vec{r},t)$ is the velocity field of the fluid and $\Phi=\Phi(\vec{r},t)$
is the newtonian gravitational potential. $P$ is the pressure of the fluid. Equation \ref{eq:continuuity} is the
continuuity equation which tells us that the density changes with the flow of
mass. Equation \ref{eq:eulereq}, which is the Euler equation, describes the
behaviour of the fluid under the influence of an external force. The last
equation, equation \ref{eq:poisson}, is the Poisson equation which relates the
gravitational potential to the density field. These equations are solvable in
the limit $\vert\delta\vert \ll 1$, and can then be used to derive an expression
for relatively small density contrasts, which can be used to describe the
evolution of density perturbations on large scales (SJekk opp om dette er
riktig.). We can model small perturbations to first order by substituing $\rho =
\rho_0 + \delta \rho$, $\vec{v} =\vec{v_0} + \delta \vec{v}$, $\vec{v} =\vec{v_0}
+ \delta \vec{v}$, $P = P_o + \delta P$ and $\Phi = \Phi_0 +\delta\Phi$. Here
$\delta$ is used to assign small perturbations to the physical quantities and should not be
confused with the previously defined density contrast. The velocity also has to
take into account the expansion of the universe in addition to the peculiar
velocity of the particles giving $\vec{v} = H(t)\vec{r} + \vec{v}_{pec}$. After a
thorough calculation, and transforming the solution to comoving coordinates $\vec{x}$, (Kanskje sitere dette? Den er veldig lang) one will arrive at the following expression for the
evolution of the density contrast
\begin{equation}
    \frac{\partial^2 \delta}{\partial t^2} + 2H(t) \frac{d \delta}{dt}=4\pi G\bar{\rho}\delta.
\end{equation}
This equation only contains time derivatives, and all the coefficients does not
depend on $\vec{x}$. Therefore $\delta(\vec{x}, t)$ can then be expressed a spatial and time
dependent expression on the form 
\begin{equation}
    \delta(\vec{x}, t) = D(t)g(\vec{x}).
\end{equation}
Here $g(\vec{x})$ is an arbitrary function of the spatial comoving coordinate
$\vec{x}$, and $D(t)$ satisfies
\begin{equation}
    \frac{\partial^2 D}{\partial t^2} + 2H(t) \frac{d D}{dt}=4\pi G\bar{\rho}D.
\end{equation}
This equation has two solutions. One solution is strictly increasing and one is
strictly decreasing. As time evolves the decreasing solution will become negligible
and the increasing solution will dominate. This solution, denoted as $D_+(t)$,
is called the growth factor and determines the amplitude of structure growth as
a function of time. Since $t\propto a$, one can then model the growth factor for
a given cosmological model. We will lastly define the function
\begin{equation}
    f(a) \equiv \frac{d log D_+}{d log a},
\end{equation}
to quantify the relationship between the growth of structure and the expansion
of the universe.

\subsection{Two point correlation functions and the matter power spectrum.}

The matter power spectrum $P(k)$ provides a statistical description of the
distribution of matter in the universe.
